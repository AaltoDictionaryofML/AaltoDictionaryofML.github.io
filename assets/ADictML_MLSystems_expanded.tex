%% ------------------------------------------------------------------
%% AUTO-GENERATED by FlattenGlossary.py
%% Source: /Users/junga1/AaltoDictionaryofML.github.io/ADictML_MLSystems.tex
%% Repo root: /Users/junga1/AaltoDictionaryofML.github.io
%% ------------------------------------------------------------------

\newglossaryentry{mlsystem}
{name={machine learning system (ML system)},
	description={An machine learning (ML) system\index{machine learning system (ML system)} 
		consists of computational devices that can gather and store data, 
		execute algorithms, and exchange information via communication networks. 
		Examples of the exchanged information include data or updates of 
		model parameters. Conceptually, an machine learning (ML) system is distinct from 
		an machine learning (ML) algorithm: an algorithm specifies 
		the abstract computational procedure (e.g., an optimization method), 
		while the system specifies how this procedure is realized in practice
		\cite{SSipser2013,KnuthAlgoBook,tanenbaum2006modern}.
		Examples of algorithms executed by devices within an machine learning system (ML system) include
		gradient-based methods for solving empirical risk minimization (ERM) problems.
		\\
		See also: machine learning (ML), device. },
 	type=mlsystems, 
 	first={machine learning system (ML system)}, 
 	text={ML system},
 	firstplural={machine learning systems (ML systems)},
 	plural={ML systems}
}


\newglossaryentry{automaton}
{name={automaton},
 description={An\index{automaton} automaton is a mathematical representation of a computing device 
              whose behaviour is described by a set of internal states, a memory 
			  structure, and a state-transition rule. Formally, an automaton consists 
			  of a state space, a set of admissible memory configurations, 
              and a transition function that specifies how the current state 
			  and memory are updated in response to inputs \cite{Sipser2013}. 
			  The notion of an automaton is useful for the anaylsis of algorithms, such 
			  as those used in machine learning (ML) methods \cite{Cormen:2022aa}. 
              Collections of interacting automata can be used to study distributed algorithm, 
			  where each automaton represents a device that executes local computations and 
			  communicates with other devices \cite{LynchDistributedAlgorithms,ParallelDistrBook}.}, 
  first = {automaton}, 
  type=mlsystems,
  plural ={automata}, 
  text = {automaton} 
}


\newglossaryentry{flsystem}
{name={federated learning system (FL system)},
	description={A federated learning (FL) system\index{federated learning system (FL system)} is a 
		distributed machine learning system (ML system) in which multiple computational devices 
		collaborate to train machine learning (ML) models without sharing their raw local data.
		An federated learning (FL) system is characterized by a communication network that specifies 
		which devices can exchange information. Conceptually, an federated learning (FL) system is 
		distinct from an federated learning (FL) algorithm \cite{DistributedSystems}. The system 
		specifies the participating entities, their interconnections, and execution constraints, while the algorithm 
		specifies the update rules for local and global model parameters \cite{KnuthAlgoBook,LynchDistributedAlgorithms}.
		Typical information exchanged in an federated learning (FL) system includes 
		model parameters or gradient information, but not raw data.
		See also: federated learning (FL), machine learning system (ML system), Federated Learning Network (FL network), algorithm.},
 	type=mlsystems,
 	first={federated learning system (FL system)},
 	text={FL system},
 	firstplural={federated learning systems (FL systems)},
 	plural={FL systems}
}


\newglossaryentry{cloudcomputing}
{name={cloud computing},
	description={Cloud computing\index{cloud computing} is a computing paradigm in which 
		computational resources such as processing, storage, and networking are provided 
		as on-demand services over a communication network
		\cite{DistributedSystems,ComerCloudComputing2021}. 
		In machine learning (ML), cloud computing systems are commonly used to host large datasets 
		and to execute machine learning (ML) algorithms. In contrast to federated learning systems (FL systems), cloud 
		computing typically centralizes data and computation within provider-managed 
		data centers.
		See also: machine learning system (ML system), federated learning system (FL system).},
	type=mlsystems,
	first={cloud computing},
	text={cloud computing}
}

\newglossaryentry{mlaas}
{name={machine learning as a service (MLaaS)},
	description={MLaaS\index{machine learning as a service (MLaaS)} refers to a 
	             cloud computing service model in which machine learning 
				 capabilities are provided to users via standardized network interfaces. 
				 In this model, the cloud provider manages the underlying computing 
				 infrastructure, data storage, and software platforms, while users 
				 access functionality such as model training and inference 
				 without direct control over physical resources \cite{ComerCloudComputing2021}.
				 \\
		See also: machine learning system (ML system), cloud computing.},
	type=mlsystems,
	first={machine learning as a service (MLaaS)},
	text={MLaaS}
}


\newglossaryentry{dynamicalsystem}
{name={dynamical system},
	description={A dynamical system\index{dynamical system} is an abstract system whose 
				 output depends on an internal state that evolves over time 
				 according to a state-update rule \cite{StrogatzBook}. In discrete time, 
				 a dynamical system is commonly described by an iteration of the form 
				 ${\bf s}^{(t+1)} =\mathcal{F}{\bf s}^{(t)}$, where ${\bf s}^{(t)}$ 
				 denotes the state at time $t$ and $\mathcal{F}$ is 
				 a state-transition map. In continuous time, dynamical systems 
				 are described by differential equations.},
	first={dynamical system},
	text={dynamical system},
	type=mlsystems,
	plural={dynamical systems}, 
	firstplural={dynamical systems}
}


%edge computing

%federated learning (system view, not algorithmic variants)

%parameter server

%client–server architecture

%orchestration

%asynchronous updates

%data locality

%deployment

%inference service

%scalability

%resource heterogeneity