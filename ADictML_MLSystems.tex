\newglossaryentry{mlsystem}
{name={machine learning system (ML system)},
	description={An \gls{ml} system\index{machine learning system (ML system)} 
		consists of computational \glspl{device} that can gather and store \gls{data}, 
		execute \glspl{algorithm}, and exchange information via communication networks. 
		Examples of the exchanged information include \gls{data} or updates of 
		\glspl{modelparam}. Conceptually, an \gls{ml} system is distinct from 
		an \gls{ml} \gls{algorithm}, i.e., an \gls{algorithm} specifies 
		the abstract computational procedure (e.g., an \gls{optmethod}), 
		while the system specifies how this procedure is realized in practice
		\cite{Sipser2013}, \cite{KnuthAlgoBook}, \cite{tanenbaum2006modern}.
		Examples of \glspl{algorithm} executed by \glspl{device} within an \gls{ml} system include
		\glspl{gdmethod} for solving \gls{erm} problems.
		\\
		See also: \gls{ml}, \gls{device}. },
 	type=mlsystems, 
 	first={machine learning system (ML system)}, 
 	text={ML system},
 	firstplural={machine learning systems (ML systems)},
 	plural={ML systems}
}

\newglossaryentry{automaton}
{name={automaton},
	description={An\index{automaton} automaton is a mathematical representation of a computing \gls{device} 
		whose behavior is described by a set of internal \glspl{state}, a memory 
		structure, and a \gls{state}-transition rule. Formally, an automaton consists 
		of a \gls{statespace}, a set of admissible memory configurations, 
		and a transition \gls{function} that specifies how the current \gls{state} 
		and memory are updated in response to inputs \cite{Sipser2013}. 
		The notion of an automaton is useful for the analysis of \glspl{algorithm}, such 
		as those used in \gls{ml} methods \cite{Cormen:2022aa}. 
		Collections of interacting automata can be used to study \glspl{distributedalgorithm}, 
		where each automaton represents a \gls{device} that executes local computations and 
		communicates with other \glspl{device} \cite{ParallelDistrBook}, \cite{LynchDistributedAlgorithms}.
		\\
		See also: \gls{device}, \gls{state}, \gls{algorithm}. }, 
	first={automaton}, 
	type=mlsystems,
	firstplural={automata}, 
	plural={automata}, 
	text={automaton} 
}

\newglossaryentry{flsystem}
{name={federated learning system (FL system)},
	description={An \gls{fl} system\index{federated learning system (FL system)} is a 
		distributed \gls{mlsystem} in which multiple computational \glspl{device} 
		collaborate to train \gls{ml} \glspl{model} without sharing their raw local \gls{data}.
		An \gls{fl} system is characterized by a communication network that specifies 
		which \glspl{device} can exchange information. Conceptually, an \gls{fl} system is 
		distinct from an \gls{fl} \gls{algorithm} \cite{DistributedSystems}. The system 
		specifies the participating entities, their interconnections, and execution constraints, while the \gls{algorithm} 
		specifies the update rules for local and global \glspl{modelparam} \cite{LynchDistributedAlgorithms}, \cite{KnuthAlgoBook}.
		Typical information exchanged in an \gls{fl} system includes 
		\glspl{modelparam} or \gls{gradient} information, but not raw \gls{data}.
		\\
		See also: \gls{fl}, \gls{mlsystem}, \gls{algorithm}, \gls{flnetwork}.},
 	type=mlsystems,
 	first={federated learning system (FL system)},
 	text={FL system},
 	firstplural={federated learning systems (FL systems)},
 	plural={FL systems}
}

\newglossaryentry{checkpoint}
{name={checkpoint},
 description={A checkpoint\index{checkpoint} is a saved representation 
   			  of the state of a running computation \cite{Elnozahy2002}.
 			  In an \gls{mlsystem}, a checkpoint typically includes the current \glspl{modelparam} 
			  during \gls{model} \gls{training} \cite{Abadi2016}},
		first={checkpoint},
		plural={checkpoints},
		firstplural={checkpoints},
		type=mlsystems,
		plural={checkpoints}
}

\newglossaryentry{checkpointing}
{name={checkpointing},
 description={Checkpointing\index{checkpointing} is a fault-tolerance mechanism
			   that periodically creates \glspl{checkpoint} by saving the state 
			   of a running computation to persistent storage. Checkpointing is essential 
			   for fault-tolerant execution on revocable resources
			   such as \glspl{spotinstance} \cite{Elnozahy2002,Mazzucco2011}.},
 first={checkpointing},
 type=mlsystems,
 text={checkpointing}
}


\newglossaryentry{spotinstance}
{name={spot instance},
 description={A spot instance\index{spot instance} is a 
              type of \gls{cloudcomputing} service that 
			  provides computational resources at a reduced cost but 
			  without guarantees on availability \cite{Mazzucco2011}.
			  In particular, a spot instance may be revoked by the provider 
			  at any time which requires to use checkpointing \cite{Khatua2013}.}, 
 first={spot instance},
 type=mlsystems,
 text={spot instance},
 firstplural={spot instances},
 plural={spot instances}
}

\newglossaryentry{cloudcomputing}
{name={cloud computing},
	description={Cloud computing\index{cloud computing} is a computing paradigm in which 
		computational resources such as processing, storage, and networking are provided 
		as on-demand services over a communication network
		\cite{DistributedSystems}, \cite{ComerCloudComputing2021}, \cite{Armbrust2010}. 
		In \gls{ml}, cloud computing systems are commonly used to host large \glspl{dataset} 
		and to execute \gls{ml} \glspl{algorithm}. In contrast to \glspl{flsystem}, cloud 
		computing typically centralizes \gls{data} and computation within provider-managed 
		\gls{data} centers.
		\\
		See also: \gls{flsystem}, \gls{mlsystem}.},
	type=mlsystems,
	first={cloud computing},
	text={cloud computing}
}

\newglossaryentry{mlaas}
{name={machine learning as a service (MLaaS)},
	description={MLaaS\index{machine learning as a service (MLaaS)} refers to a 
		\gls{cloudcomputing} service \gls{model} in which \gls{ml} 
		capabilities are provided to users via standardized network interfaces. 
		In this \gls{model}, the cloud provider manages the underlying computing 
		infrastructure, \gls{data} storage, and software platforms, while users 
		access functionality such as \gls{model} \gls{training} and \gls{inference} 
		without direct control over physical resources \cite{ComerCloudComputing2021}.
				 \\
		See also: \gls{cloudcomputing}, \gls{mlsystem}.},
	type=mlsystems,
	first={machine learning as a service (MLaaS)},
	text={MLaaS}
}

\newglossaryentry{dynamicalsystem}
{name={dynamical system},
	description={A dynamical system\index{dynamical system} is an abstract system whose 
		\gls{output} depends on an internal \gls{state} that evolves over time 
		according to a \gls{state}-update rule \cite{StrogatzBook}. In discrete time, 
		a dynamical system is commonly described by an \gls{iteration} of the form 
		$\vs^{(\timeidx+1)} =\fixedpointop \vs^{(\timeidx)}$, where $\vs^{(\timeidx)}$ 
		denotes the \gls{state} at time $\timeidx$ and $\fixedpointop$ is 
		a \gls{state}-transition \gls{map}. In continuous time, dynamical systems 
		are described by differential equations.
		\\
		See also: \gls{output}, \gls{state}. },
	first={dynamical system},
	text={dynamical system},
	type=mlsystems,
	plural={dynamical systems}, 
	firstplural={dynamical systems}
}

\newglossaryentry{edgecomputing}
{name={edge computing},
	description={Edge computing\index{edge computing} refers to the placement of
		computation and \gls{data} storage close to the sources of \gls{data} generation,
		such as sensors, \glspl{mobiledevice}, or embedded systems, rather than in
		centralized \gls{data} centers \cite{Shi2016}. 
		In \gls{ml}, edge computing supports low-latency \gls{inference} and reduced
		communication by executing parts of \gls{ml} \glspl{algorithm} on or near
		\gls{data}-generating \glspl{device} \cite{Nguyen2019}. 
		\\
		See also: \gls{cloudcomputing}, \gls{flsystem}, \gls{mlsystem}.},
	type=mlsystems,
	first={edge computing},
	text={edge computing}
}

\newglossaryentry{edgedevice}
{name={edge device},
	description={An edge \gls{device}\index{edge device} operates at or 
		near the edge of a communication network \cite{Shi2016}. The term edge refers 
		to the periphery of the network, where \gls{data} is produced and first processed.
		In \gls{ml} and, in particular, in \gls{fl}, an edge \gls{device} 
		typically corresponds to a node of an \gls{flnetwork}. Each edge \gls{device} 
		stores local \gls{data} and implements parts of the \gls{mlpipeline}, 
		such as \gls{data} \gls{preprocessing}, \gls{localmodel} \gls{training}, 
		or \gls{inference} \cite{Hashash2021}. 
		\\ 
		See also: \gls{device}, \gls{edgecomputing}.},
	type=mlsystems,
	first={edge device},
	plural={edge devices},
	firstplural={edge devices},
	text={edge device}
}

\newglossaryentry{preprocessing}
{name={preprocessing},
	description={Preprocessing\index{preprocessing} refers to the set of operations
		applied to raw \gls{data} before they are fed into an \gls{ml} \gls{algorithm} \cite{hastie01statisticallearning}. 
		The goal of preprocessing is to transform the \gls{data} into a form
		that is more suitable for follow-up stages of an \gls{mlpipeline}. 
		Typical preprocessing steps include cleaning corrupted or missing values,
		normalizing or scaling \glspl{feature}, or encoding categorical variables \cite{MLPipeline}. 
		\\
		See also: \gls{data}, \gls{mlpipeline}, \gls{feature}.},
	first={preprocessing},
	type=mlsystems,
	text={preprocessing}
}

\newglossaryentry{mlpipeline}
{name={machine learning pipeline (ML pipeline)},
	description={The term ML pipeline\index{machine learning pipeline (ML pipeline)} refers to a composition (i.e., concatenation)
		of several \glspl{function} within an \gls{mlsystem}. The individual \glspl{function} 
		include \gls{data} \gls{preprocessing}, \gls{featlearn}, \gls{model} \gls{training}, and \gls{inference}. 
		By combining them, an \gls{mlsystem} turns raw \gls{data} into \glspl{prediction} \cite{MLPipeline}.  
		\\
		See also: \gls{function}, \gls{mlsystem}.},
	first={machine learning pipeline (ML pipeline)},
	type=mlsystems,
	text={ML pipeline},
	plural={machine learning pipelines (ML pipelines)},
	firstplural={ML pipelines}
}

\newglossaryentry{mobiledevice}
{name={mobile device},
	description={A mobile \gls{device}\index{mobile device} is a portable computing
		\gls{device} equipped with computational, storage, sensing, and wireless
		communication capabilities \cite{Lane2015}, \cite{Tsoukas2024}. 
		Examples of mobile \glspl{device} include smartphones, tablets, 
		or wearables. Mobile \glspl{device} can act as \gls{data} sources 
		and provide computational infrastructure for \glspl{edgecomputing} 
		or \glspl{flsystem}.
		\\
		See also: \gls{edgecomputing}, \gls{flsystem}, \gls{mlsystem}.},
	type=mlsystems,
	first={mobile device},
	plural={mobile devices},
	firstplural={mobile devices},
	text={mobile device}
}

%federated learning (system view, not algorithmic variants)

%parameter server

%client–server architecture

%orchestration

%asynchronous updates

%data locality

%deployment

%inference service

%scalability

%resource heterogeneity