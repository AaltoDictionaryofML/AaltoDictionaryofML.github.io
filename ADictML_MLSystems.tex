\newglossaryentry{mlsystem}
{name={machine learning system (ML system)},
	description={An \gls{ml} system\index{machine learning system (ML system)} 
		consists of computational \glspl{device} that can gather and store \gls{data}, 
		execute \glspl{algorithm}, and exchange information via communication networks. 
		Examples of the exchanged information include \gls{data} or updates of 
		\glspl{modelparam}. Conceptually, an \gls{ml} system is distinct from 
		an \gls{ml} \gls{algorithm}: an \gls{algorithm} specifies 
		the abstract computational procedure (e.g., an \gls{optmethod}), 
		while the system specifies how this procedure is realized in practice
		\cite{SSipser2013,KnuthAlgoBook,tanenbaum2006modern}.
		Examples of \glspl{algorithm} executed by \glspl{device} within an \gls{mlsystem} include
		\glspl{gdmethod} for solving \gls{erm} problems.
		\\
		See also: \gls{ml}, \gls{device}. },
 	type=mlsystems, 
 	first={machine learning system (ML system)}, 
 	text={ML system},
 	firstplural={machine learning systems (ML systems)},
 	plural={ML systems}
}


\newglossaryentry{automaton}
{name={automaton},
 description={An\index{automaton} automaton is a mathematical representation of a computing \gls{device} 
              whose behaviour is described by a set of internal \glspl{state}, a memory 
			  structure, and a \gls{state}-transition rule. Formally, an automaton consists 
			  of a \gls{state} space, a set of admissible memory configurations, 
              and a transition \gls{function} that specifies how the current \gls{state} 
			  and memory are updated in response to inputs \cite{Sipser2013}. 
			  The notion of an automaton is useful for the anaylsis of \glspl{algorithm}, such 
			  as those used in \gls{ml} methods \cite{Cormen:2022aa}. 
              Collections of interacting automata can be used to study \gls{distributedalgorithm}, 
			  where each automaton represents a \gls{device} that executes local computations and 
			  communicates with other \glspl{device} \cite{LynchDistributedAlgorithms,ParallelDistrBook}.}, 
  first = {automaton}, 
  type=mlsystems,
  plural ={automata}, 
  text = {automaton} 
}


\newglossaryentry{flsystem}
{name={federated learning system (FL system)},
	description={A \gls{fl} system\index{federated learning system (FL system)} is a 
		distributed \gls{mlsystem} in which multiple computational \glspl{device} 
		collaborate to train \gls{ml} models without sharing their raw local \gls{data}.
		An \gls{fl} system is characterized by a communication network that specifies 
		which \glspl{device} can exchange information. Conceptually, an \gls{fl} system is 
		distinct from an \gls{fl} \gls{algorithm} \cite{DistributedSystems}. The system 
		specifies the participating entities, their interconnections, and execution constraints, while the \gls{algorithm} 
		specifies the update rules for local and global \glspl{modelparam} \cite{KnuthAlgoBook,LynchDistributedAlgorithms}.
		Typical information exchanged in an \gls{fl} system includes 
		\glspl{modelparam} or \gls{gradient} information, but not raw \gls{data}.
		See also: \gls{fl}, \gls{mlsystem}, \gls{flnetwork}, \gls{algorithm}.},
 	type=mlsystems,
 	first={federated learning system (FL system)},
 	text={FL system},
 	firstplural={federated learning systems (FL systems)},
 	plural={FL systems}
}


\newglossaryentry{cloudcomputing}
{name={cloud computing},
	description={Cloud computing\index{cloud computing} is a computing paradigm in which 
		computational resources such as processing, storage, and networking are provided 
		as on-demand services over a communication network
		\cite{DistributedSystems,ComerCloudComputing2021}. 
		In \gls{ml}, cloud computing systems are commonly used to host large \glspl{dataset} 
		and to execute \gls{ml} \glspl{algorithm}. In contrast to \glspl{flsystem}, cloud 
		computing typically centralizes \gls{data} and computation within provider-managed 
		\gls{data} centers.
		See also: \gls{mlsystem}, \gls{flsystem}.},
	type=mlsystems,
	first={cloud computing},
	text={cloud computing}
}

\newglossaryentry{mlaas}
{name={machine learning as a service (MLaaS)},
	description={MLaaS\index{machine learning as a service (MLaaS)} refers to a 
	             \gls{cloudcomputing} service model in which machine learning 
				 capabilities are provided to users via standardized network interfaces. 
				 In this model, the cloud provider manages the underlying computing 
				 infrastructure, \gls{data} storage, and software platforms, while users 
				 access functionality such as \gls{model} \gls{training} and inference 
				 without direct control over physical resources \cite{ComerCloudComputing2021}.
				 \\
		See also: \gls{mlsystem}, \gls{cloudcomputing}.},
	type=mlsystems,
	first={machine learning as a service (MLaaS)},
	text={MLaaS}
}




%edge computing

%federated learning (system view, not algorithmic variants)

%parameter server

%client–server architecture

%orchestration

%asynchronous updates

%data locality

%deployment

%inference service

%scalability

%resource heterogeneity