\newglossaryentry{trustAI}
{name={\foreignlanguage{greek}{αξιόπιστη τεχνητή νοημοσύνη (αξιόπιστη ΤΝ)}},
	description={\foreignlanguage{greek}{Εκτός από τις} \glsentryuservi{compasp} \foreignlanguage{greek}{και τις} 
		\glsentryuservi{statasp}, \foreignlanguage{greek}{μία τρίτη κύρια διάσταση σχεδιασμού μεθόδων}  
		\glsentryuserii{ml} \foreignlanguage{greek}{είναι η αξιοπιστία 
		τους}\index{\foreignlanguage{greek}{αξιόπιστη τεχνητή νοημοσύνη (αξιόπιστη ΤΝ)}} 
		\cite{pfau2024engineeringtrustworthyaideveloper}. 
		\foreignlanguage{greek}{Η Ευρωπαϊκή Ένωση (ΕΕ) έχει διατυπώσει επτά βασικές απαιτήσεις για αξιόπιστη}  
		\glsentryuseriii{ai} (trustworthy artificial intelligence - trustworthy AI) \foreignlanguage{greek}{(οι οποίες 
		συνήθως χτίζουν πάνω σε μεθόδους} \glsentryuserii{ml}) \cite{ALTAIEU}: 
		\begin{enumerate}[label=\arabic*)]
			\item \foreignlanguage{greek}{Ανθρώπινη παρέμβαση και εποπτεία·}
			\item \foreignlanguage{greek}{Τεχνική} \glsentryuseri{robustness} \foreignlanguage{greek}{και ασφάλεια·}
			\item \foreignlanguage{greek}{Ιδιωτικότητα και διακυβέρνηση των} \glsentryuserii{data}·
			\item \glsentryuseriv{transparency}·
			\item \foreignlanguage{greek}{Διαφορετικότητα, απαγόρευση των διακρίσεων και δικαιοσύνη·}
			\item \foreignlanguage{greek}{Κοινωνιακή και περιβαλλοντική ευημερία·}
			\item \foreignlanguage{greek}{Λογοδοσία.}
		\end{enumerate}
		\foreignlanguage{greek}{Βλέπε επίσης:} \gls{compasp}, \gls{statasp}, \gls{ml}, \gls{ai}, \gls{robustness}, \gls{data}, \gls{transparency}.},
	first={\foreignlanguage{greek}{αξιόπιστη τεχνητή νοημοσύνη (αξιόπιστη ΤΝ)}},
	text={\foreignlanguage{greek}{αξιόπιστη ΤΝ}},
	type=regulation,
	user1={\foreignlanguage{greek}{αξιόπιστη ΤΝ}}, %nominative
  	user2={\foreignlanguage{greek}{αξιόπιστης ΤΝ}}, %genitive 
	user3={\foreignlanguage{greek}{αξιόπιστη ΤΝ}} %accusative
}

\newglossaryentry{automateddecisionmaking}
{name={\foreignlanguage{greek}{αυτοματοποιημένη λήψη αποφάσεων}},
	description={\foreignlanguage{greek}{Η αυτοματοποιημένη λήψη 
		αποφάσεων}\index{\foreignlanguage{greek}{αυτοματοποιημένη λήψη αποφάσεων}}
		(automated decision-making) \foreignlanguage{greek}{αναφέρεται στις εφαρμογές} 
		\glsentryuserii{ml} \foreignlanguage{greek}{που χρησιμοποιούν} \glsentryuservi{prediction} 
		\foreignlanguage{greek}{που παράγονται από ένα εκπαιδευμένο} \glsentryuseriii{model} 
		\foreignlanguage{greek}{απευθείας (δηλαδή χωρίς ανθρώπινη παρέμβαση) για τη
		λήψη αποφάσεων που επηρεάζουν άτομα. Με βάση τον} \glsentryuseriii{gdpr}, 
		\foreignlanguage{greek}{τα άτομα έχουν το δικάιωμα να μην υπόκεινται σε αποφάσεις 
		που βασίζονται μόνο σε αυτοματοποιημένη επεξεργασία, όταν αυτές οι αποφάσεις  
		έχουν νομικές ή όμοια σημαντικές επιδράσεις, εκτός εάν υλοποιούνται κατάλληλα μέτρα 
		προστασίας (π.χ. ανθρώπινη εποπτεία, δυνατότητα αμφισβήτησης, ή ρητή συγκατάθεση).}
		\\
		\foreignlanguage{greek}{Βλέπε επίσης:} \gls{ml}, \gls{prediction}, \gls{model}, \gls{gdpr}. }, 
	first={\foreignlanguage{greek}{αυτοματοποιημένη λήψη αποφάσεων}},
	text={\foreignlanguage{greek}{αυτοματοποιημένη λήψη αποφάσεων}},
	type=regulation,
	user1={\foreignlanguage{greek}{αυτοματοποιημένη λήψη αποφάσεων}}, %nominative
	user2={\foreignlanguage{greek}{αυτοματοποιημένης λήψης αποφάσεων}}, %genitive
	user3={\foreignlanguage{greek}{αυτοματοποιημένη λήψη αποφάσεων}}, %accusative
}

\newglossaryentry{gdpr}
{name={\foreignlanguage{greek}{γενικός κανονισμός για την προστασία δεδομένων (ΓΚΠΔ)}},
	description={\foreignlanguage{greek}{Ο ΓΚΠΔ}\index{\foreignlanguage{greek}{γενικός κανονισμός για την προστασία δεδομένων (ΓΚΠΔ)}} 
		(general data protection regulation - GDPR) \foreignlanguage{greek}{θεσπίστηκε από την ΕΕ και τέθηκε σε ισχύ 
		από τις 25 Μαΐου 2018} \cite{GDPR2016}. \foreignlanguage{greek}{Διαφυλάσσει την ιδιωτικότητα και τα δικαιώματα} 
		\glsentryuserii{data} \foreignlanguage{greek}{των ατόμων στην ΕΕ. Ο ΓΚΠΔ έχει σημαντικές επιπτώσεις 
		για το πώς συλλέγονται} \glsentryuseriii{data}, \foreignlanguage{greek}{πώς αποθηκεύονται, 
		και πώς χρησιμοποιούνται στις εφαρμογές} \glsentryuserii{ml}. \foreignlanguage{greek}{Βασικές 
		διατάξεις περιλαμβάνουν τα εξής}:
		\begin{itemize}
			\item \glsentryuseriv{dataminprinc}: \foreignlanguage{greek}{Τα} \glsentryuseriv{mlsystem}
				\foreignlanguage{greek}{θα πρέπει να χρησιμοποιούν μόνο την απαραίτητη ποσότητα} 
				\glsentryuserii{personaldata} \foreignlanguage{greek}{για τον σκοπό τους.} 
			\item \glsentryuseriv{transparency} \foreignlanguage{greek}{και} \glsentryuseri{explainability}: 
				\foreignlanguage{greek}{Τα} \glsentryuseriv{mlsystem} \foreignlanguage{greek}{θα πρέπει να επιτρέπουν 
				στους χρήστες τους να κατανοούν πώς τα συστήματα παίρνουν αποφάσεις που επηρεάζουν τους χρήστες.} 
			\item \foreignlanguage{greek}{Δικαιώματα των υποκειμένων των} \glsentryuserii{data}: \foreignlanguage{greek}{Οι 
				χρήστες θα πρέπει να έχουν την ευκαιρία να έχουν πρόσβαση, να διορθώνουν, και να διαγράφουν 
				τα} \glsentryuseriii{personaldata} \foreignlanguage{greek}{τους, καθώς και να αντιτίθενται 
				στην} \glsentryuseriii{automateddecisionmaking} \foreignlanguage{greek}{και στην} \glsentryuseriii{profiling}. 
			\item \foreignlanguage{greek}{Λογοδοσία: Οι οργανισμοί πρέπει να εξασφαλίζουν την εύρωστη ασφάλεια} 
				\glsentryuserii{data} \foreignlanguage{greek}{και να αποδεικνύουν συμμόρφωση μέσω τεκμηρίων 
				και τακτικών ελέγχων.} 
		\end{itemize}
		\foreignlanguage{greek}{Βλέπε επίσης:} \gls{data}, \gls{ml}, \gls{dataminprinc}, \gls{mlsystem}, \gls{personaldata}, 
		\gls{transparency}, \gls{explainability}, \gls{automateddecisionmaking}, \gls{profiling}.}, 
	first={\foreignlanguage{greek}{γενικός κανονισμός για την προστασία δεδομένων (ΓΚΠΔ)}},
	text={\foreignlanguage{greek}{ΓΚΠΔ}},
	type=regulation,
	user1={\foreignlanguage{greek}{ΓΚΠΔ}}, %nominative
    	user2={\foreignlanguage{greek}{ΓΚΠΔ}}, %genitive
	user3={\foreignlanguage{greek}{ΓΚΠΔ}} %accusative 
}

\newglossaryentry{transparency}
{name={\foreignlanguage{greek}{διαφάνεια}},
	description={\foreignlanguage{greek}{Η διαφάνεια}\index{\foreignlanguage{greek}{διαφάνεια}} 
		\foreignlanguage{greek}{είναι μία θεμελιώδης απαίτηση για} 
		\glsentryuseriii{trustAI} \cite{HLEGTrustworhtyAI}. \foreignlanguage{greek}{Στο πλαίσιο μεθόδων} 
		\glsentryuserii{ml}, \foreignlanguage{greek}{η διαφάνεια χρησιμοποιείται συχνά εναλλακτικά με την} 
		\glsentryuseriii{explainability} \cite{JunXML2020}, \cite{gallese2023ai}. 
		\foreignlanguage{greek}{Ωστόσο, στο ευρύτερο πεδίο} \glsentryuserv{aisystem}, 
		\foreignlanguage{greek}{η διαφάνεια επεκτείνεται πέρα από την} \glsentryuseriii{explainability} 
		\foreignlanguage{greek}{και περιλαμβάνει την παροχή πληροφοριών σχετικά με τους περιορισμούς,  
		την αξιοπιστία, και την επιθυμητή χρήση του συστήματος. Σε συστήματα ιατρικής διάγνωσης,
		η διαφάνεια απαιτεί τη γνωστοποίηση του επιπέδου εμπιστοσύνης για τις} \glsentryuservi{prediction} 
		\foreignlanguage{greek}{που παραδίδονται από ένα εκπαιδευμένο} \glsentryuseriii{model}. 
		\foreignlanguage{greek}{Στην πιστωτική ικανότητα, οι αποφάσεις δανεισμού που βασίζονται στην} 
		\glsentryuseriii{ai} \foreignlanguage{greek}{θα πρέπει να συνοδεύονται από} \glsentryuservi{explanation} 
		\foreignlanguage{greek}{παραγόντων που συμβάλλουν, όπως το επίπεδο εισοδήματος ή το πιστωτικό 
		ιστορικό. Αυτές οι} \glsentryuseriv{explanation} \foreignlanguage{greek}{επιτρέπουν τους ανθρώπους
		(π.χ. έναν αιτούντα δανείου) να κατανοήσουν και να αμφισβητήσουν αυτοματοποιημένες αποφάσεις.  
		Κάποιες μέθοδοι} \glsentryuserii{ml} \foreignlanguage{greek}{προσφέρουν εγγενώς διαφάνεια. 
		Για παράδειγμα, η} \glsentryuseri{logreg} \foreignlanguage{greek}{παρέχει ένα ποσοτικό}
		\glsentryuseriii{measure} \foreignlanguage{greek}{της αξιοπιστίας της} \glsentryuserii{classification} 
		\foreignlanguage{greek}{μέσω της τιμής $|\hypothesis(\featurevec)|$. Ένα ακόμα παράδειγμα αποτελούν τα}
		\glsentryuservi{decisiontree}, \foreignlanguage{greek}{καθώς επιτρέπουν κανόνες αποφάσεων 
		που είναι αναγνώσιμοι από άνθρωπο} \cite{rudin2019stop}.
		\foreignlanguage{greek}{Η διαφάνεια επίσης απαιτεί μία σαφή ένδειξη όταν ένας χρήστης αλληλεπιδρά  
		με ένα} \glsentryuseriii{aisystem}. \foreignlanguage{greek}{Για παράδειγμα, τα} chatbots
		\foreignlanguage{greek}{που λειτουργούν με} \glsentryuseriii{ai} \foreignlanguage{greek}{θα πρέπει να 
		ειδοποιούν τους χρήστες ότι αλληλεπι\-δρούν με ένα αυτοματοποιημένo σύστημα και όχι με άνθρωπο. 
		Επιπλέον, η διαφάνεια συμπεριλαμβάνει περιεκτική τεκμηρίωση που περιγράφει λεπτομερώς 
		τον σκοπό και τις επιλογές σχεδιασμού που αποτελούν τη βάση του} \glsentryuserii{aisystem}. 
		\foreignlanguage{greek}{Για παράδειγμα, τα φύλλα δεδομένων} \glsentryuserv{model} 
		\cite{DatasheetData2021} \foreignlanguage{greek}{και οι κάρτες} \glsentryuserv{aisystem}  
		\cite{10.1145/3287560.3287596} \foreignlanguage{greek}{βοηθούν τους επαγγελματίες να κατανοήσουν 
		τις περιπτώσεις επιθυμητής χρήσης και τους περιορισμούς ενός} \glsentryuserii{aisystem} \cite{Shahriari2017}. 
		\\
		\foreignlanguage{greek}{Βλέπε επίσης:} \gls{trustAI}, \gls{ml}, \gls{explainability}, \gls{aisystem}, \gls{prediction}, 
		\gls{model}, \gls{ai}, \gls{explanation}, \gls{logreg}, \gls{measure}, \gls{classification}, \gls{decisiontree}. },
	first={\foreignlanguage{greek}{διαφάνεια}}, 
	text={\foreignlanguage{greek}{διαφάνεια}},
	type=regulation,
	user1={\foreignlanguage{greek}{διαφάνεια}}, %nominative
    	user2={\foreignlanguage{greek}{διαφάνειας}}, %genitive 
	user3={\foreignlanguage{greek}{διαφάνεια}}, %accusative
	user4={\foreignlanguage{greek}{Διαφάνεια}} %nominativecapital
}

\newglossaryentry{profiling}
{name={\foreignlanguage{greek}{κατάρτιση προφίλ}},
	description={\foreignlanguage{greek}{Η κατάρτιση προφίλ}\index{\foreignlanguage{greek}{κατάρτιση προφίλ}} 
		(profiling) \foreignlanguage{greek}{στοχεύει στην αναγνώριση μοτίβων και στην εξαγωγή συμπερασμάτων 
		σχετικά με άτομα βάσει των} \glsentryuserii{data} \foreignlanguage{greek}{τους. Οι τεχνικές 
		κατάρτισης προφίλ χρησιμοποιούν μεθόδους} \glsentryuserii{ml} \foreignlanguage{greek}{για να 
		προβλέψουν την επίδοση ατόμων στη δουλειά, την οικονομική τους κατάσταση, την υγεία ή 
		τις προσωπικές τους προτιμήσεις. Η κατάρτιση προφίλ είναι καίρια για τη στοχευμένη διαφήμιση, 
		την πιστωτική ικανότητα, τον εντοπισμό απάτης, και τις εξατομικευμένες υπηρεσίες. Ο}
		\glsentryuseri{gdpr} \foreignlanguage{greek}{επιβάλλει αυστηρές απαιτήσεις σε οργανισμούς  
		που ασχολούνται με δραστηριότητες κατάρτισης προφιλ ώστε να εξασφαλίζεται η προστασία των  
		δικαιωμάτων των ατόμων} \cite{GDPR2016}. 
		\\
		\foreignlanguage{greek}{Βλέπε επίσης:} \gls{data}, \gls{ml}, \gls{gdpr}. },
	first={\foreignlanguage{greek}{κατάρτιση προφίλ}},
	text={\foreignlanguage{greek}{κατάρτιση προφίλ}},
	type=regulation,
	user1={\foreignlanguage{greek}{κατάρτιση προφίλ}}, %nominative
	user2={\foreignlanguage{greek}{κατάρτισης προφίλ}}, %genitive
	user3={\foreignlanguage{greek}{κατάρτιση προφίλ}} %accusative
}

\newglossaryentry{deepfake}
{name={\foreignlanguage{greek}{προϊόν βαθυπαραποίησης}},
	description={\foreignlanguage{greek}{Τα προϊόντα βαθυπαραποίησης} (deep 
		fakes)\index{\foreignlanguage{greek}{προϊόν βαθυπαραποίησης}} \foreignlanguage{greek}{εί\-ναι 
		συνθετικά μέσα που παράγονται ή τροποποιούνται σημαντικά από ένα}
		\glsentryuseriii{aisystem}, \foreignlanguage{greek}{έτσι ώστε να φαίνεται ψευδώς ότι 
		απεικονίζουν ένα πραγματικό πρόσωπο, αντικείμενο, ή} \glsentryuseriii{event}. 
		\foreignlanguage{greek}{Τα προϊόντα βαθυπαραποίησης παράγονται συνήθως 
		με τη χρήση παραγωγικών μεθόδων, οι οποίες εκπαιδεύονται να μιμούνται οπτικά, 
		ακουστικά, ή οπτικοακουστικά χαρακτηριστικά πραγματικών} \glsentryuserii{data}. 
		\foreignlanguage{greek}{Από άποψη συστήματος, τα προϊόντα βαθυπαραποίησης
		χαρακτηρίζονται από μία σκόπιμη ασυμβατικότητα
            	μεταξύ του παρατηρήσιμου περιεχομένου και της πραγματικής προέλευσης, 
		γεγονός που μπορεί να οδηγήσει σε απάτη, παραπληροφόρηση, ή χειραγώγηση.} 
		\\
             	%CITE relavant legal sources.	
		\foreignlanguage{greek}{Βλέπε επίσης:} \gls{aisystem}, \gls{event}, \gls{data}.	},
 	first={\foreignlanguage{greek}{προϊόν βαθυπαραποίησης}},
 	text={\foreignlanguage{greek}{προϊόν βαθυπαραποίησης}},
	type=regulation,
	user1={\foreignlanguage{greek}{προϊόν βαθυπαραποίησης}}, %nominative
	user2={\foreignlanguage{greek}{προϊόντος βαθυπαραποίησης}}, %genitive
	user3={\foreignlanguage{greek}{προϊόν βαθυπαραποίησης}}, %accusative
	user4={\foreignlanguage{greek}{προϊόντα βαθυπαραποίησης}}, %nominativepl
	user5={\foreignlanguage{greek}{προϊόντων βαθυπαραποίησης}}, %genitivepl 
	user6={\foreignlanguage{greek}{προϊόντα βαθυπαραποίησης}} %accusativepl
}

\newglossaryentry{personaldata}
{name={\foreignlanguage{greek}{προσωπικά δεδομένα}},
	description={\foreignlanguage{greek}{Προσωπικά} \glsentryuseri{data} (personal data) 
		\foreignlanguage{greek}{είναι κάθε πληροφορία}\index{\foreignlanguage{greek}{προσωπικά δεδομένα}} 
		\foreignlanguage{greek}{που σχετίζεται με ένα ταυτοποιημένο ή ταυτοποιήσιμο φυσικό
		πρόσωπο (δηλαδή το υποκείμενο των} \glsentryuserii{data}). \foreignlanguage{greek}{Ένα 
		φυσικό πρόσωπο είναι ταυτοποιήσιμο αν μπορεί να ταυτοποιηθεί, άμεσα ή έμμεσα, 
		συγκεκριμένα με αναφορά σε ένα αναγνωριστικό όπως όνομα, αριθμό ταυτοποίησης}, 
		\glsentryuseriii{data} \foreignlanguage{greek}{τοποθεσίας, διαδικτυακό αναγνωριστικό, 
		ή έναν ή περισσότερους παράγοντες συγκεκριμένα για τη σωματική, φυσιολογική,
		γενετική, διανοητική, οικονομική, πολιτιστική, ή κοινωνική ταυτότητα αυτού του
		ατόμου} \cite{GDPR2016}. \foreignlanguage{greek}{Στα} \glsentryuservi{mlsystem}, 
		\foreignlanguage{greek}{προσωπικά} \glsentryuseri{data} \foreignlanguage{greek}{μπορεί 
		να εμφανίζονται στα} \glsentryuseriii{data} \glsentryuserii{training}, 
		\foreignlanguage{greek}{στις εισόδους του} \glsentryuserii{model}, \foreignlanguage{greek}{στις 
		ενδιάμεσες αναπαραστάσεις (π.χ.} \glsentryuservi{featurevec} \foreignlanguage{greek}{ή εμφυτεύσεις),
		ή στις} \glsentryuservi{output} \foreignlanguage{greek}{του} \glsentryuserii{model}, 
		\foreignlanguage{greek}{εφόσον οι πληροφορίες σχετίζονται με ένα ταυτοποιήσιμο φυσικό πρόσωπο.
		Ο Κανονισμός της ΕΕ για την ΤΝ} (Artificial Intelligence Act - AI Act) \foreignlanguage{greek}{δεν εισάγει 
		έναν ξεχωριστο ορισμό για τα προσωπικά} \glsentryuseriii{data}· \foreignlanguage{greek}{αντ΄ αυτού,
		κάθε φορά που ένα} \glsentryuseri{aisystem} \foreignlanguage{greek}{επεξεργάζεται προσωπικά} \glsentryuseriii{data}, 
		\foreignlanguage{greek}{εφαρμόζονται πλήρως ο ορισμός και οι υποχρεώσεις του} \glsentryuserii{gdpr}. 
		\\
		\foreignlanguage{greek}{Βλέπε επίσης:} \gls{data}, \gls{mlsystem}, \gls{training}, \gls{model}, 
		\gls{featurevec}, \gls{output}, \gls{aisystem}, \gls{gdpr}. },
  	first={\foreignlanguage{greek}{προσωπικά δεδομένα}},
	text={\foreignlanguage{greek}{προσωπικά δεδομένα}},
	type=regulation,
	user1={\foreignlanguage{greek}{προσωπικά δεδομένα}}, %nominativepl
	user2={\foreignlanguage{greek}{προσωπικών δεδομένων}}, %genitivepl 
	user3={\foreignlanguage{greek}{προσωπικά δεδομένα}} %accusativepl
}

\newglossaryentry{aisystem}
{name={\foreignlanguage{greek}{σύστημα τεχνητής νοημοσύνης (σύστημα ΤΝ)}},
	description={\foreignlanguage{greek}{Ο Κανονισμός της ΕΕ για την 
		ΤΝ}\index{\foreignlanguage{greek}{σύστημα τεχνητής νοημοσύνης (σύστημα ΤΝ)}} 
		(AI Act) \cite{AIact} \foreignlanguage{greek}{ορίζει ένα σύστημα} \glsentryuserii{ai} 
		(artificial intelligence system - AI system) \foreignlanguage{greek}{ως ένα σύστημα βασισμένο σε μηχανή
		που έχει σχεδιαστεί να λειτουργεί με μεταβαλλόμενα επίπεδα αυτονομίας και που
		μπορεί να επιδεικνύει προσαρμοστικότητα (π.χ. επανεκπαίδευση} \glsentryuserii{model})
		\foreignlanguage{greek}{μετά την ανάπτυξή του. Τα συστήματα} \glsentryuserii{ai} 
		\foreignlanguage{greek}{υπολογίζουν} \glsentryuservi{prediction} \foreignlanguage{greek}{που 
		μπορούν να επηρεάσουν περιβάλλοντα ή αποφάσεις} \cite{EUAIAct2024}. 
		\foreignlanguage{greek}{Σε συμφωνία με αυτόν τον ορισμό, οι κανονιστικές
		υποχρεώσεις και οι κατηγορίες κινδύνου εφαρμόζονται στο επίπεδο του συστήματος} 
		\glsentryuserii{ai} \foreignlanguage{greek}{και όχι στο επίπεδο μεμονωμένων}
		\glsentryuserv{model} \foreignlanguage{greek}{ή} \glsentryuserv{algorithm}.  
		\foreignlanguage{greek}{Η εξέταση σε επίπεδο συστήματος τονίζει ότι οι ιδιότητες όπως η} 
		\glsentryuseri{robustness}, \foreignlanguage{greek}{η δικαιοσύνη, και η} \glsentryuseri{transparency} 
		\foreignlanguage{greek}{προκύπτουν από την αλληλεπίδραση των} \glsentryuserv{model}, 
		\foreignlanguage{greek}{των} \glsentryuserii{data}, \foreignlanguage{greek}{και του λειτουργικού 
		πλαισίου, και όχι από μεμονωμένες συνιστώσες.} 
		\\
		\foreignlanguage{greek}{Βλέπε επίσης:} \gls{ai}, \gls{model}, \gls{prediction}, \gls{algorithm},\gls{robustness}, 
		\gls{transparency}, \gls{data}. },
 	first={\foreignlanguage{greek}{σύστημα τεχνητής νοημοσύνης (σύστημα ΤΝ)}},
 	text={\foreignlanguage{greek}{σύστημα ΤΝ}},
 	type=regulation,
	user1={\foreignlanguage{greek}{σύστημα ΤΝ}}, %nominative
	user2={\foreignlanguage{greek}{συστήματος ΤΝ}}, %genitive
	user3={\foreignlanguage{greek}{σύστημα ΤΝ}}, %accusative
	user4={\foreignlanguage{greek}{συστήματα ΤΝ}}, %nominativepl
	user5={\foreignlanguage{greek}{συστημάτων ΤΝ}}, %genitivepl 
	user6={\foreignlanguage{greek}{συστήματα ΤΝ}} %accusativepl
}

\newglossaryentry{highriskaisystem}
{name={\foreignlanguage{greek}{σύστημα τεχνητής νοημοσύνης υψηλού κινδύνου (σύστημα ΤΝ υψηλού κινδύνου)}},
	description={\foreignlanguage{greek}{Ένα υποσύνολο 
		των}\index{\foreignlanguage{greek}{σύστημα τεχνητής νοημοσύνης υψηλού κινδύνου (σύστημα ΤΝ υψηλού κινδύνου)}} 
		\glsentryuserv{aisystem} \foreignlanguage{greek}{ταξινομείται ως υψηλού κινδύνου λόγω 
		της δυνατότητάς του να επηρεάζει σημαντικά την ασφάλεια, τα θεμελιώδη 
		δικαιώματα, ή τις κοινωνικές λειτουργίες ζωτικής σημασίας. Τα} \glsentryuseriv{aisystem} 
		\foreignlanguage{greek}{υψηλού κινδύνου} (high-risk artificial intelligence system - high-risk AI system) 
		\foreignlanguage{greek}{υπόκεινται σε αυστηρές κανονιστικές απαιτήσεις με βάση τον Κανονισμό 
		της ΕΕ για την ΤΝ} (AI Act), \foreignlanguage{greek}{οι οποίες περιλαμβάνουν αξιολογήσεις
		συμμόρφωσης, διαχείριση κινδύνου, υποχρεώσεις} \glsentryuserii{transparency}, 
		\foreignlanguage{greek}{και παρακολούθηση μετά τη διάθεση στην αγορά} \cite{EUAIAct2024}. 
		\foreignlanguage{greek}{Παραδείγματα} \glsentryuserv{aisystem} \foreignlanguage{greek}{υψηλού κινδύνου 
		περιλαμβάνουν εκείνα που χρησιμοποιούνται σε υποδομές ζωτικής σημασίας, στην εκπαίδευση, 
		στην απασχόληση, στην επιβολή του νόμου, και στη βιομετρική ταυτοποίηση.} 
		\\
		\foreignlanguage{greek}{Βλέπε επίσης:} \gls{aisystem}, \gls{transparency}. },
 	first={\foreignlanguage{greek}{σύστημα τεχνητής νοημοσύνης υψηλού κινδύνου (σύστημα ΤΝ υψηλού κινδύνου)}},
	text={\foreignlanguage{greek}{σύστημα ΤΝ υψηλού κινδύνου}},
	type=regulation,
	user1={\foreignlanguage{greek}{σύστημα ΤΝ υψηλού κινδύνου}}, %nominative
	user2={\foreignlanguage{greek}{συστήματος ΤΝ υψηλού κινδύνου}}, %genitive
	user3={\foreignlanguage{greek}{σύστημα ΤΝ υψηλού κινδύνου}}, %accusative
	user4={\foreignlanguage{greek}{συστήματα ΤΝ υψηλού κινδύνου}}, %nominativepl
	user5={\foreignlanguage{greek}{συστημάτων ΤΝ υψηλού κινδύνου}}, %genitivepl 
	user6={\foreignlanguage{greek}{συστήματα ΤΝ υψηλού κινδύνου}} %accusativepl
}

\newglossaryentry{shap}
{name={SHapley Additive exPlanations (SHAP)},
	description={TBC.\index{SHapley Additive exPlanations (SHAP)} },
	first={SHapley Additive exPlanations (SHAP)},
	type=regulation, 
	text={SHAP}
}

% \newglossaryentry{provider}
% {name={provider},
% 	description={TBC (AI TERMINOLOGY)\index{provider}.},
% 	first={provider},
%     type=regulation, 
% 	plural={providers}, 
% 	firstplural={providers},
% 	text={provider}
% }

% \newglossaryentry{deployer}
% {name={deployer},
% 	description={TBC (AI TERMINOLOGY)\index{deployer}.},
% 	first={deployer},
% 	plural={deployers}, 
%     type=regulation, 
% 	firstplural={deployers},
% 	text={deployer}
% }

% \newglossaryentry{authrepresentative}
% {name={authorized representative},
% 	description={TBC (AI TERMINOLOGY)\index{authorized representative}.},
% 	first={authorized representative},
% 	plural={authorized representatives}, 
%     type=regulation, 
% 	firstplural={authorized representatives},
% 	text={authorized representative}
% }

% \newglossaryentry{importer}
% {name={importer},
% 	description={TBC (AI TERMINOLOGY)\index{importer}.},
% 	first={importer},
% 	plural={importers}, 
%     type=regulation, 
% 	firstplural={importers},
% 	text={importer}
% }

% \newglossaryentry{distributor}
% {name={distributor},
% 	description={TBC (AI TERMINOLOGY)\index{distributor}.},
% 	first={distributor},
% 	plural={distributors},
%     type=regulation,  
% 	firstplural={distributors},
% 	text={distributor}
% }

% \newglossaryentry{placingonmarket}
% {name={placing on the market},
% 	description={TBC (AI TERMINOLOGY)\index{placing on the market}.},
% 	first={placing on the market},
%     type=regulation, 
% 	text={placing on the market}
% }

% \newglossaryentry{availableonmarket}
% {name={making available on the market},
% 	description={TBC (AI TERMINOLOGY)\index{making available on the market}.},
% 	first={making available on the market},
%     type=regulation, 
% 	text={making available on the market}
% }

% \newglossaryentry{intoservice}
% {name={putting into service},
% 	description={TBC (AI TERMINOLOGY)\index{putting into service}.},
% 	first={putting into service},
%     type=regulation, 
% 	text={putting into service}
% }

% \newglossaryentry{intendedpurpose}
% {name={intended purpose},
% 	description={TBC (AI TERMINOLOGY)\index{intended purpose}.},
% 	first={intended purpose},
% 	plural={intended purposes}, 
% 	firstplural={intended purposes},
%     type=regulation, 
% 	text={intended purpose}
% }

% \newglossaryentry{foreseeablemisuse}
% {name={reasonably foreseeable misuse},
% 	description={TBC (AI TERMINOLOGY)\index{reasonably foreseeable misuse}.},
% 	first={reasonably foreseeable misuse},
% 	plural={reasonably foreseeable misuses}, 
% 	firstplural={reasonably foreseeable misuses},
%     type=regulation, 
% 	text={reasonably foreseeable misuse}
% }

% \newglossaryentry{safetycomponent}
% {name={safety component},
% 	description={TBC (AI TERMINOLOGY)\index{safety component}.},
% 	first={safety component},
% 	plural={safety components}, 
% 	firstplural={safety components},
%     type=regulation, 
% 	text={safety component}
% }

% \newglossaryentry{useinstructions}
% {name={instructions for use},
% 	description={TBC (AI TERMINOLOGY)\index{instructions for use}.},
% 	first={instructions for use},
%     type=regulation, 
% 	text={instructions for use}
% }

% \newglossaryentry{airecall}
% {name={recall of an artificial intelligence system (recall of an AI system)},
% 	description={TBC (AI TERMINOLOGY)\index{recall of an artificial intelligence system (recall of an AI system)}.},
% 	first={recall of an artificial intelligence system (recall of an AI system)},
%     type=regulation, 
% 	text={recall of an AI system}
% }

% \newglossaryentry{aiwithdrawal}
% {name={withdrawal of an artificial intelligence system (withdrawal of an AI system)},
% 	description={TBC (AI TERMINOLOGY)\index{withdrawal of an artificial intelligence system (withdrawal of an AI system)}.},
% 	first={withdrawal of an artificial intelligence system (withdrawal of an AI system)},
%     type=regulation, 
% 	text={withdrawal of an AI system}
% }

% \newglossaryentry{aiperformance}
% {name={performance of an artificial intelligence system (performance of an AI system)},
% 	description={TBC (AI TERMINOLOGY)\index{performance of an artificial intelligence system (performance of an AI system)}.},
% 	first={performance of an artificial intelligence system (performance of an AI system)},
%     type=regulation, 
% 	text={performance of an AI system}
% }

% \newglossaryentry{notifyingauth}
% {name={notifying authority},
% 	description={TBC (AI TERMINOLOGY)\index{notifying authority}.},
% 	first={notifying authority},
% 	plural={notifying authorities}, 
% 	firstplural={notifying authorities},
%     type=regulation, 
% 	text={notifying authority}
% }

% \newglossaryentry{conformityassessment}
% {name={conformity assessment},
% 	description={TBC (AI TERMINOLOGY)\index{conformity assessment}.},
% 	first={conformity assessment},
% 	plural={conformity assessments}, 
% 	firstplural={conformity assessments},
%     type=regulation, 
% 	text={conformity assessment}
% }

% \newglossaryentry{conformityassessmentbody}
% {name={conformity assessment body},
% 	description={TBC (AI TERMINOLOGY)\index{conformity assessment body}.},
% 	first={conformity assessment body},
% 	plural={conformity assessment bodies}, 
% 	firstplural={conformity assessment bodies},
%     type=regulation, 
% 	text={conformity assessment body}
% }

% \newglossaryentry{notifiedbody}
% {name={notified body},
% 	description={TBC (AI TERMINOLOGY)\index{notified body}.},
% 	first={notified body},
% 	plural={notified bodies}, 
% 	firstplural={notified bodies},
%     type=regulation, 
% 	text={notified body}
% }

% \newglossaryentry{substantialmod}
% {name={substantial modification},
% 	description={TBC (AI TERMINOLOGY)\index{substantial modification}.},
% 	first={substantial modification},
% 	plural={substantial modifications}, 
% 	firstplural={substantial modifications},
%     type=regulation, 
% 	text={substantial modification}
% }

% \newglossaryentry{cemarking}
% {name={CE marking},
% 	description={TBC (AI TERMINOLOGY)\index{CE marking}.},
% 	first={CE marking},
% 	plural={CE markings}, 
% 	firstplural={CE markings},
%     type=regulation, 
% 	text={CE marking }
% }

% \newglossaryentry{postmonitoring}
% {name={post-market monitoring system},
% 	description={TBC (AI TERMINOLOGY)\index{post-market monitoring system}.},
% 	first={post-market monitoring system},
% 	plural={post-market monitoring systems}, 
% 	firstplural={post-market monitoring systems},
%     type=regulation, 
% 	text={post-market monitoring system}
% }

% \newglossaryentry{surveillanceauth}
% {name={market surveillance authority},
% 	description={TBC (AI TERMINOLOGY)\index{market surveillance authority}.},
% 	first={market surveillance authority},
% 	plural={market surveillance authorities}, 
% 	firstplural={market surveillance authorities},
%     type=regulation, 
% 	text={market surveillance authority}
% }

% \newglossaryentry{harmonizedstandard}
% {name={harmonized standard},
% 	description={TBC (AI TERMINOLOGY)\index{harmonized standard}.},
% 	first={harmonized standard},
% 	plural={harmonized standards}, 
%     type=regulation, 
% 	firstplural={harmonized standards},
% 	text={harmonized standard}
% }

% \newglossaryentry{commonspecification}
% {name={common specification},
% 	description={TBC (AI TERMINOLOGY)\index{common specification}.},
% 	first={common specification},
% 	plural={common specifications}, 
% 	firstplural={common specifications},
%     type=regulation, 
% 	text={common specification}
% }

% \newglossaryentry{biometricdata}
% {name={biometric data},
% 	description={TBC (AI TERMINOLOGY)\index{biometric data}.},
% 	first={biometric data},
%     type=regulation, 
% 	text={biometric data}
% }

% \newglossaryentry{biometricidentification}
% {name={biometric identification},
% 	description={TBC (AI TERMINOLOGY)\index{biometric identification}.},
% 	first={biometric identification},
%     type=regulation, 
% 	text={biometric identification}
% }

% \newglossaryentry{biometricverification}
% {name={biometric verification},
% 	description={TBC (AI TERMINOLOGY)\index{biometric verification}.},
% 	first={biometric verification},
%     type=regulation, 
% 	text={biometric verification}
% }

% \newglossaryentry{specialpersonaldata}
% {name={special categories of personal data},
% 	description={TBC (AI TERMINOLOGY)\index{special categories of personal data}.},
% 	first={special categories of personal data},
%     type=regulation, 
% 	text={special categories of personal data}
% }

% \newglossaryentry{sensitiveoperdata}
% {name={sensitive operational data},
% 	description={TBC (AI TERMINOLOGY)\index{sensitive operational data}.},
% 	first={sensitive operational data},
%     type=regulation, 
% 	text={sensitive operational data}
% }

% \newglossaryentry{emotionrecog}
% {name={emotion recognition system},
% 	description={TBC (AI TERMINOLOGY)\index{emotion recognition system}.},
% 	first={emotion recognition system},
% 	plural={emotion recognition systems}, 
% 	firstplural={emotion recognition systems},
%     type=regulation, 
% 	text={emotion recognition system}
% }

% \newglossaryentry{biometriccateg}
% {name={biometric categorization system},
% 	description={TBC (AI TERMINOLOGY)\index{biometric categorization system}.},
% 	first={biometric categorization system},
% 	plural={biometric categorization systems}, 
% 	firstplural={biometric categorization systems},
%     type=regulation, 
% 	text={biometric categorization system}
% }

% \newglossaryentry{remotebiometricident}
% {name={remote biometric identification system},
% 	description={TBC (AI TERMINOLOGY)\index{remote biometric identification system}.},
% 	first={remote biometric identification system},
% 	plural={remote biometric identification systems}, 
% 	firstplural={remote biometric identification systems},
%     type=regulation, 
% 	text={remote biometric identification system}
% }

% \newglossaryentry{realtimeremotebiometricident}
% {name={real-time remote biometric identification system},
% 	description={TBC (AI TERMINOLOGY)\index{real-time remote biometric identification system}.},
% 	first={real-time remote biometric identification system},
% 	plural={real-time remote biometric identification systems}, 
% 	firstplural={real-time remote biometric identification systems},
%     type=regulation, 
% 	text={real-time remote biometric identification system}
% }

% \newglossaryentry{postremotebiometricident}
% {name={post-remote biometric identification system},
% 	description={TBC (AI TERMINOLOGY)\index{post-remote biometric identification system}.},
% 	first={post-remote biometric identification system},
% 	plural={post-remote biometric identification systems}, 
% 	firstplural={post-remote biometric identification systems},
%     type=regulation, 
% 	text={post-remote biometric identification system}
% }

% \newglossaryentry{publicaccessspace}
% {name={publicly accessible space},
% 	description={TBC (AI TERMINOLOGY)\index{publicly accessible space}.},
% 	first={publicly accessible space},
% 	plural={publicly accessible spaces}, 
% 	firstplural={publicly accessible spaces},
%     type=regulation, 
% 	text={publicly accessible space}
% }

% \newglossaryentry{lawauth}
% {name={law enforcement authority},
% 	description={TBC (AI TERMINOLOGY)\index{law enforcement authority}.},
% 	first={law enforcement authority},
% 	plural={law enforcement authorities}, 
% 	firstplural={law enforcement authorities},
%     type=regulation, 
% 	text={law enforcement authority}
% }

% \newglossaryentry{lawenforcement}
% {name={law enforcement},
% 	description={TBC (AI TERMINOLOGY)\index{law enforcement}.},
% 	first={law enforcement},
% 	plural={law enforcements}, 
% 	firstplural={law enforcements},
%     type=regulation, 
% 	text={law enforcement}
% }

% \newglossaryentry{aioffice}
% {name={Artificial Intelligence Office (AI Office)},
% 	description={TBC (AI TERMINOLOGY)\index{Artificial Intelligence Office (AI Office)}.},
% 	first={Artificial Intelligence Office (AI Office)},
%     type=regulation, 
% 	text={AI Office}
% }

% \newglossaryentry{nationalauth}
% {name={national competent authority},
% 	description={TBC (AI TERMINOLOGY)\index{national competent authority}.},
% 	first={national competent authority},
% 	plural={national competent authorities}, 
% 	firstplural={national competent authorities},
%     type=regulation, 
% 	text={national competent authority}
% }

% \newglossaryentry{seriousincident}
% {name={serious incident},
% 	description={TBC (AI TERMINOLOGY)\index{serious incident}.},
% 	first={serious incident},
% 	plural={serious incidents}, 
% 	firstplural={serious incidents},
%     type=regulation, 
% 	text={serious incident}
% }

%\newglossaryentry{realtestplan}
%{name={real-world testing plan},
%	description={TBC (AI TERMINOLOGY)\index{real-world testing plan}.},
%	first={real-world testing plan},
%    	type=regulation, 
%	text={real-world testing plan}
%}

% \newglossaryentry{sandboxplan}
% {name={sandbox plan},
% 	description={TBC (AI TERMINOLOGY)\index{sandbox plan}.},
% 	first={sandbox plan},
% 	plural={sandbox plans}, 
% 	firstplural={sandbox plans},
%     type=regulation, 
% 	text={sandbox plan}
% }

% \newglossaryentry{aisandbox}
% {name={artificial intelligence regulatory sandbox (AI regulatory sandbox)},
% 	description={TBC (AI TERMINOLOGY)\index{artificial intelligence regulatory sandbox (AI regulatory sandbox)}.},
% 	first={artificial intelligence regulatory sandbox (AI regulatory sandbox)},
% 	plural={AI regulatory sandboxes}, 
% 	firstplural={artificial intelligence regulatory sandboxes (AI regulatory sandboxes)},
%     type=regulation, 
% 	text={AI regulatory sandbox}
% }

% \newglossaryentry{ailiteracy}
% {name={artificial intelligence literacy (AI literacy)},
% 	description={TBC (AI TERMINOLOGY)\index{artificial intelligence literacy (AI literacy)}.},
% 	first={artificial intelligence literacy (AI literacy)},
%     type=regulation, 
% 	text={AI literacy}
% }

% \newglossaryentry{testinginrealcond}
% {name={testing in real-world conditions},
% 	description={TBC (AI TERMINOLOGY)\index{testing in real-world conditions}.},
% 	first={testing in real-world conditions},
%     type=regulation, 
% 	text={testing in real-world conditions}
% }

% \newglossaryentry{subject}
% {name={subject},
% 	description={TBC (AI TERMINOLOGY)\index{subject}.},
% 	first={subject},
%     type=regulation, 
% 	plural={subjects}, 
% 	firstplural={subjects},
% 	text={subject}
% }

% \newglossaryentry{informedconsent}
% {name={informed consent},
% 	description={TBC (AI TERMINOLOGY)\index{informed consent}.},
% 	first={informed consent},
% 	plural={informed consents}, 
%     type=regulation, 
% 	firstplural={informed consents},
% 	text={informed consent}
% }

% \newglossaryentry{widinfringe}
% {name={widespread infringement},
% 	description={TBC (AI TERMINOLOGY)\index{widespread infringement}.},
% 	first={widespread infringement},
%     type=regulation, 
% 	text={widespread infringement}
% }

% \newglossaryentry{criticalinfrastructure}
% {name={critical infrastructure},
% 	description={TBC (AI TERMINOLOGY)\index{critical infrastructure}.},
% 	first={critical infrastructure},
% 	plural={critical infrastructures}, 
%     type=regulation, 
% 	firstplural={critical infrastructures},
% 	text={critical infrastructure}
% }

% \newglossaryentry{generalaimodel}
% {name={general-purpose artificial intelligence model (general-purpose AI model)},
% 	description={TBC (AI TERMINOLOGY)\index{general-purpose artificial intelligence model (general-purpose AI model)}.},
% 	first={general-purpose artificial intelligence model (general-purpose AI model)},
% 	plural={general-purpose AI models}, 
%     type=regulation, 
% 	firstplural={general-purpose artificial intelligence models (general-purpose AI models)},
% 	text={general-purpose AI model}
% }

% \newglossaryentry{highimpactcapab}
% {name={high-impact capabilities},
% 	description={TBC (AI TERMINOLOGY)\index{high-impact capabilities}.},
% 	first={high-impact capabilities},
%     type=regulation, 
% 	text={high-impact capabilities}
% }

% \newglossaryentry{systemicrisk}
% {name={systemic risk},
% 	description={TBC (AI TERMINOLOGY)\index{systemic risk}.},
% 	first={systemic risk},
% 	plural={systemic risks}, 
%     type=regulation, 
% 	firstplural={systemic risks},
% 	text={systemic risk}
% }

% \newglossaryentry{floatpointoper}
% {name={floating-point operation},
% 	description={TBC (AI TERMINOLOGY)\index{floating-point operation}.},
% 	first={floating-point operation},
% 	plural={floating-point operations}, 
%     type=regulation, 
% 	firstplural={floating-point operations},
% 	text={floating-point operation}
% }

% \newglossaryentry{downstreamprovider}
% {name={downstream provider},
% 	description={TBC (AI TERMINOLOGY)\index{downstream provider}.},
% 	first={downstream provider},
% 	plural={downstream providers}, 
%     type=regulation, 
% 	firstplural={downstream providers},
% 	text={downstream provider}
% }