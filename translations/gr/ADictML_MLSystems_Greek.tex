\newglossaryentry{automaton}
{name={\foreignlanguage{greek}{αυτόματο}},
	description={\foreignlanguage{greek}{Ένα αυτόματο}\index{\foreignlanguage{greek}{αυτόματο}} (automaton) 
		\foreignlanguage{greek}{είναι μία μαθηματική αναπαράσταση μίας υπολογιστικής} \glsentryuserii{device}, 
		\foreignlanguage{greek}{της οποίας η συμπεριφορά περιγράφεται από ένα σύνολο εσωτερικών} 
		\glsentryuserv{state}, \foreignlanguage{greek}{μία δομή μνήμης, και έναν κανόνα} 
		\glsentryuserii{state}-\foreignlanguage{greek}{μετάβασης. Τυπικά, ένα αυτόματο αποτελείται από έναν} 
		\glsentryuseriii{statespace}, \foreignlanguage{greek}{ένα σύνολο αποδεκτών διαρθρώσεων μνήμης, 
		και μία} \glsentryuseriii{function} \foreignlanguage{greek}{μετάβασης που προσδιορίζει το πώς η τρέχουσα} 
		\glsentryuseri{state} \foreignlanguage{greek}{και η μνήμη ενημερώνονται ως απάντηση στις εισόδους}
		\cite{Sipser2013}. \foreignlanguage{greek}{Η έννοια του αυτόματου είναι χρήσιμη για την ανάλυση}
		\glsentryuserv{algorithm}, \foreignlanguage{greek}{όπως εκείνων που χρησιμοποιούνται σε μεθόδους} 
		\glsentryuserii{ml} \cite{Cormen:2022aa}. \foreignlanguage{greek}{Συλλογές αυτομάτων που αλληλεπιδρούν 
		μπορούν να χρησιμοποιηθούν για τη μελέτη} \glspl{distributedalgorithm}, 
		\foreignlanguage{greek}{όπου κάθε αυτόματο αναπαριστά μία} \glsentryuseriii{device} \foreignlanguage{greek}{που 
		εκτελεί τοπικούς υπολογισμούς και επικοινωνεί με άλλες} \glsentryuservi{device} 
		\cite{ParallelDistrBook}, \cite{LynchDistributedAlgorithms}.
		\\
		\foreignlanguage{greek}{Βλέπε επίσης:} \gls{device}, \gls{state}, \gls{statespace}, \gls{function}, 
		\gls{algorithm}, \gls{ml}, \glspl{distributedalgorithm}. }, 
	first={\foreignlanguage{greek}{αυτόματο}},
	text={\foreignlanguage{greek}{αυτόματο}},
	type=mlsystems,
	user1={\foreignlanguage{greek}{αυτόματο}}, %nominative
	user2={\foreignlanguage{greek}{αυτόματου}}, %genitive
	user3={\foreignlanguage{greek}{αυτόματο}}, %accusative
	user4={\foreignlanguage{greek}{αυτόματα}}, %nominativepl
	user5={\foreignlanguage{greek}{αυτομάτων}}, %genitivepl 
	user6={\foreignlanguage{greek}{αυτόματα}} %accusativepl
}

\newglossaryentry{longitudinaldata}
{name={\foreignlanguage{greek}{διαχρονικά δεδομένα}},
	description={Longitudinal \gls{data}\index{longitudinal data} consist of \glspl{datapoint} 
		whose attributes are measured repeatedly over time \cite{fitzmaurice2011applied}. 
		In \gls{ml}, longitudinal \gls{data} are common in applications such 
		as healthcare, where patient measurements are taken at multiple time points \cite{Mallinckrodt:2017aa}.
		\\
		See also: \gls{data}, \gls{datapoint}. }, 
	first={\foreignlanguage{greek}{διαχρονικά δεδομένα}},
	text={\foreignlanguage{greek}{διαχρονικά δεδομένα}},
	type=mlsystems,
	user4={\foreignlanguage{greek}{διαχρονικά δεδομένα}}, %nominativepl
	user5={\foreignlanguage{greek}{διαχρονικών δεδομένων}}, %genitivepl 
	user6={\foreignlanguage{greek}{διαχρονικά δεδομένα}} %accusativepl
}

\newglossaryentry{dynamicalsystem}
{name={\foreignlanguage{greek}{δυναμικό σύστημα}},
	description={\foreignlanguage{greek}{Ένα δυναμικό σύστημα}\index{\foreignlanguage{greek}{δυναμικό σύστημα}} 
		(dynamical system) \foreignlanguage{greek}{είναι ένα αφηρημένο σύστημα, του οποίου η}  
		\glsentryuseri{output} \foreignlanguage{greek}{εξαρτάται από μία εσωτερική} \glsentryuseriii{state} 
		\foreignlanguage{greek}{που εξελίσσεται σταδιακά σύμφωνα με έναν κανόνα} 
		\glsentryuserii{state}-\foreignlanguage{greek}{ενημέρωσης} \cite{StrogatzBook}. 
		\foreignlanguage{greek}{Στον διακριτό χρόνο, ένα δυναμικό σύστημα περιγράφεται συχνά
		από μία} \glsentryuseriii{iteration} \foreignlanguage{greek}{της μορφής}  
		$\vs^{(\timeidx+1)} =\fixedpointop \vs^{(\timeidx)}$, \foreignlanguage{greek}{όπου $\vs^{(\timeidx)}$ 
		δηλώνει την} \glsentryuseriii{state} \foreignlanguage{greek}{στη χρονική στιγμή $\timeidx$ και $\fixedpointop$ 
		είναι μία} \gls{map} \glsentryuserii{state}-\foreignlanguage{greek}{μετάβασης. Στον συνεχή χρόνο, τα δυναμικά 
		συστήματα περιγράφονται από διαφορικές εξισώσεις.}
		\\
		\foreignlanguage{greek}{Βλέπε επίσης:} \gls{output}, \gls{state}, \gls{iteration}, \gls{map}. },
	first={\foreignlanguage{greek}{δυναμικό σύστημα}},
	text={\foreignlanguage{greek}{δυναμικό σύστημα}},
	type=mlsystems,
	user1={\foreignlanguage{greek}{δυναμικό σύστημα}}, %nominative
	user2={\foreignlanguage{greek}{δυναμικού συστήματος}}, %genitive
	user3={\foreignlanguage{greek}{δυναμικό σύστημα}} %accusative
}

\newglossaryentry{mobiledevice}
{name={\foreignlanguage{greek}{κινητή συσκευή}},
	description={A mobile \gls{device}\index{mobile device} is a portable computing
		\gls{device} equipped with computational, storage, sensing, and wireless
		communication capabilities \cite{Lane2015}, \cite{Tsoukas2024}. 
		Examples of mobile \glspl{device} include smartphones, tablets, 
		or wearables. Mobile \glspl{device} can act as \gls{data} sources 
		and provide computational infrastructure for \glspl{edgecomputing} 
		or \glspl{flsystem}.
		\\
		See also: \gls{edgecomputing}, \gls{flsystem}, \gls{mlsystem}.},
	first={\foreignlanguage{greek}{κινητή συσκευή}},
	text={\foreignlanguage{greek}{κινητή συσκευή}},
	type=mlsystems,
	user1={\foreignlanguage{greek}{κινητή συσκευή}}, %nominative
	user2={\foreignlanguage{greek}{κινητής συσκευής}}, %genitive
	user3={\foreignlanguage{greek}{κινητή συσκευή}} %accusative
}

\newglossaryentry{mlaas}
{name={\foreignlanguage{greek}{μηχανική μάθηση ως υπηρεσία}},
	description={\foreignlanguage{greek}{Η μηχανική μάθηση ως 
		υπηρεσία}\index{\foreignlanguage{greek}{μηχανική μάθηση ως υπηρεσία}} 
		(machine learning as a service - MLaaS) \foreignlanguage{greek}{αναφέρεται σε ένα}
		\glsentryuseriii{model} \foreignlanguage{greek}{υπηρεσίας}  \glsentryuserii{cloudcomputing} 
		\foreignlanguage{greek}{στο οποίο οι δυνατότητες} \glsentryuserii{ml} 
		\foreignlanguage{greek}{παρέχονται στους χρήστες μέσω τυποποιημένων διεπαφών δικτύου.}
		\foreignlanguage{greek}{Σε αυτό το} \glsentryuseriii{model}, \foreignlanguage{greek}{ο πάροχος 
		του νέφους διαχειρίζεται τις υποκείμενες υπολογιστικές υποδομές, την αποθήκευση}
		\glsentryuserii{data}, \foreignlanguage{greek}{και τις πλατφόρμες λογισμικού, ενώ οι χρήστες 
		έχουν πρόσβαση σε λειτουργικότητα όπως η} \glsentryuseri{training} \glsentryuserii{model} 
		\foreignlanguage{greek}{και η} \glsentryuseri{inference} \foreignlanguage{greek}{χωρίς άμεσο 
		έλεγχο των φυσικών πόρων} \cite{ComerCloudComputing2021}.
				 \\
		\foreignlanguage{greek}{Βλέπε επίσης:} \gls{cloudcomputing}, \gls{model}, \gls{ml}, \gls{data}, \gls{training}, 
		\gls{inference}, \gls{mlsystem}.},
	first={\foreignlanguage{greek}{μηχανική μάθηση ως υπηρεσία}},
	text={\foreignlanguage{greek}{μηχανική μάθηση ως υπηρεσία}},
	type=mlsystems,
	user1={\foreignlanguage{greek}{μηχανική μάθηση ως υπηρεσία}}, %nominative
	user2={\foreignlanguage{greek}{μηχανικής μάθησης ως υπηρεσίας}}, %genitive
	user3={\foreignlanguage{greek}{μηχανική μάθηση ως υπηρεσία}} %accusative
}%Υπηρεσία Μηχανικής Μάθησης

\newglossaryentry{preprocessing}
{name={\foreignlanguage{greek}{προεπεξεργασία}},
	description={Preprocessing\index{preprocessing} refers to the set of operations
		applied to raw \gls{data} before they are fed into an \gls{ml} \gls{algorithm} \cite{hastie01statisticallearning}. 
		The goal of preprocessing is to transform the \gls{data} into a form
		that is more suitable for follow-up stages of an \gls{mlpipeline}. 
		Typical preprocessing steps include cleaning corrupted or missing values,
		normalizing or scaling \glspl{feature}, or encoding categorical variables \cite{MLPipeline}. 
		\\
		See also: \gls{data}, \gls{mlpipeline}, \gls{feature}.},
	first={\foreignlanguage{greek}{προεπεξεργασία}},
	text={\foreignlanguage{greek}{προεπεξεργασία}},
	type=mlsystems,
	user1={\foreignlanguage{greek}{προεπεξεργασία}}, %nominative
	user2={\foreignlanguage{greek}{προεπεξεργασίας}}, %genitive  
	user3={\foreignlanguage{greek}{προεπεξεργασία}} %accusative
}

\newglossaryentry{edgedevice}
{name={\foreignlanguage{greek}{συσκευή ακμής}},
	description={An edge \gls{device}\index{edge device} operates at or 
		near the edge of a communication network \cite{Shi2016}. The term edge refers 
		to the periphery of the network, where \gls{data} is produced and first processed.
		In \gls{ml} and, in particular, in \gls{fl}, an edge \gls{device} 
		typically corresponds to a node of an \gls{flnetwork}. Each edge \gls{device} 
		stores local \gls{data} and implements parts of the \gls{mlpipeline}, 
		such as \gls{data} \gls{preprocessing}, \gls{localmodel} \gls{training}, 
		or \gls{inference} \cite{Hashash2021}. 
		\\ 
		See also: \gls{device}, \gls{edgecomputing}.},
	first={\foreignlanguage{greek}{συσκευή ακμής}},
	text={\foreignlanguage{greek}{συσκευή ακμής}},
	type=mlsystems,
	user1={\foreignlanguage{greek}{συσκευή ακμής}}, %nominative
	user2={\foreignlanguage{greek}{συσκευής ακμής}}, %genitive  
	user3={\foreignlanguage{greek}{συσκευή ακμής}} %accusative
}

\newglossaryentry{mlsystem}
{name={\foreignlanguage{greek}{σύστημα μηχανικής μάθησης}},
	description={\foreignlanguage{greek}{Ένα σύστημα} 
		\glsentryuserii{ml}\index{\foreignlanguage{greek}{σύστημα μηχανικής μάθησης}} 
		(machine learning system - ML system) \foreignlanguage{greek}{αποτελείται από υπολογιστικές} 
		\glsentryuservi{device} \foreignlanguage{greek}{που μπορούν να συλλέγουν και να 
		αποθηκεύουν} \glsentryuseriii{data}, \foreignlanguage{greek}{να εκτελούν} 
		\glsentryuservi{algorithm}, \foreignlanguage{greek}{και να ανταλλάσσουν πληροφορίες 
		μέσω δικτύων επικοινωνίας. Παραδείγματα των ανταλλασσόμενων πληροφοριών 
		περιλαμβάνουν} \glsentryuseriii{data} \foreignlanguage{greek}{ή ενημερώσεις 
		των} \glsentryuserv{modelparam}. 
		\foreignlanguage{greek}{Εννοιολογικά, ένα σύστημα} \glsentryuserii{ml} \foreignlanguage{greek}{είναι 
		διαφορετικό από έναν} \glsentryuseriii{algorithm} \glsentryuserii{ml}, \foreignlanguage{greek}{δηλαδή 
		ένας} \glsentryuseri{algorithm} \foreignlanguage{greek}{προσδιορίζει την αφηρημένη υπολογιστική 
		διαδικασία (π.χ. μία} \glsentryuseriii{optmethod}), \foreignlanguage{greek}{ενώ το σύστημα προσδιορίζει 
		το πώς αυτή η διαδικασία υλοποιείται στην πράξη}
		\cite{Sipser2013}, \cite{KnuthAlgoBook}, \cite{tanenbaum2006modern}.
		\foreignlanguage{greek}{Παραδείγματα} \glsentryuserv{algorithm} \foreignlanguage{greek}{που 
		εκτελούνται από} \glsentryuservi{device} \foreignlanguage{greek}{εντός ενός συστήματος} \glsentryuserii{ml} 
		\foreignlanguage{greek}{περιλαμβάνουν} \glsentryuservi{gdmethod} \foreignlanguage{greek}{για την επίλυση 
		προβλημάτων} \glsentryuserii{erm}.
		\\
		\foreignlanguage{greek}{Βλέπε επίσης:} \gls{ml}, \gls{device}, \gls{data}, \gls{algorithm}, \gls{modelparam}, 
		\gls{optmethod}, \gls{gdmethod}, \gls{erm}. },
 	first={\foreignlanguage{greek}{σύστημα μηχανικής μάθησης}}, 
 	text={\foreignlanguage{greek}{σύστημα μηχανικής μάθησης}},
	type=mlsystems,
	user1={\foreignlanguage{greek}{σύστημα μηχανικής μάθησης}}, %nominative
	user2={\foreignlanguage{greek}{συστήματος μηχανικής μάθησης}}, %genitive
	user3={\foreignlanguage{greek}{σύστημα μηχανικής μάθησης}}, %accusative
	user4={\foreignlanguage{greek}{συστήματα μηχανικής μάθησης}}, %nominativepl
	user5={\foreignlanguage{greek}{συστημάτων μηχανικής μάθησης}}, %genitivepl 
	user6={\foreignlanguage{greek}{συστήματα μηχανικής μάθησης}} %accusativepl
}

\newglossaryentry{flsystem}
{name={\foreignlanguage{greek}{σύστημα ομοσπονδιακής μάθησης}},
	description={\foreignlanguage{greek}{Ένα σύστημα} \glsentryuserii{fl}\index{federated learning system (FL system)} 
		\foreignlanguage{greek}{είναι ένα κατανεμημένο} \glsentryuseri{mlsystem} \foreignlanguage{greek}{στο 
		οποίο πολλαπλές υπολογιστικές} \glsentryuseriv{device} \foreignlanguage{greek}{συνεργάζονται για την 
		εκπαίδευση} \glsentryuserv{model} \glsentryuserii{ml} \foreignlanguage{greek}{χωρίς να διαμοιράζονται 
		τα ακατέργαστα τοπικά} \glsentryuseriii{data} \foreignlanguage{greek}{τους. Ένα σύστημα} \glsentryuserii{fl} 
		\foreignlanguage{greek}{χαρακτηρίζεται από ένα δίκτυο επικοινωνίας που προσδιορίζει ποιες}
		\glsentryuseriv{device} \foreignlanguage{greek}{μπορούν να ανταλλάσσουν πληροφορίες. Εννοιολογικά, ένα 
		σύστημα} \glsentryuserii{fl} \foreignlanguage{greek}{είναι διαφορετικό από έναν} 
		\glsentryuseriii{algorithm} \glsentryuserii{fl} \cite{DistributedSystems}. \foreignlanguage{greek}{Το σύστημα 
		προσδιορίζει τις οντότητες που συμμετέχουν, τις διασυνδέσεις τους, και τους περιορισμούς εκτέλεσης, 
		ενώ ο} \glsentryuseri{algorithm} \foreignlanguage{greek}{προσδιορίζει τους κανόνες ενημέρωσης για τις τοπικές 
		και καθολικές} \glsentryuservi{modelparam} \cite{LynchDistributedAlgorithms}, \cite{KnuthAlgoBook}.
		\foreignlanguage{greek}{Τυπικές πληροφορίες που ανταλλάσσονται σε ένα σύστημα} \glsentryuserii{fl} 
		\foreignlanguage{greek}{περιλαμβάνουν} \glsentryuservi{modelparam} \foreignlanguage{greek}{ή πληροφορίες} 
		\glsentryuserii{gradient}, \foreignlanguage{greek}{αλλά όχι ακατέργαστα} \glsentryuseriii{data}.
		\\
		\foreignlanguage{greek}{Βλέπε επίσης:}  \gls{fl}, \gls{mlsystem}, \gls{device}, \gls{ml} \gls{model}, \gls{data}, 
		\gls{algorithm}, \gls{modelparam}, \gls{gradient}, \gls{flnetwork}.},
 	first={\foreignlanguage{greek}{σύστημα ομοσπονδιακής μάθησης}},
 	text={\foreignlanguage{greek}{σύστημα ομοσπονδιακής μάθησης}},
 	type=mlsystems,
	user1={\foreignlanguage{greek}{σύστημα ομοσπονδιακής μάθησης}}, %nominative
	user2={\foreignlanguage{greek}{συστήματος ομοσπονδιακής μάθησης}}, %genitive
	user3={\foreignlanguage{greek}{σύστημα ομοσπονδιακής μάθησης}}, %accusative
	user4={\foreignlanguage{greek}{συστήματα ομοσπονδιακής μάθησης}}, %nominativepl
	user5={\foreignlanguage{greek}{συστημάτων ομοσπονδιακής μάθησης}}, %genitivepl 
	user6={\foreignlanguage{greek}{συστήματα ομοσπονδιακής μάθησης}} %accusativepl
}

\newglossaryentry{edgecomputing}
{name={\foreignlanguage{greek}{υπολογιστική ακμής}},
	description={Edge computing\index{edge computing} refers to the placement of
		computation and \gls{data} storage close to the sources of \gls{data} generation,
		such as sensors, \glspl{mobiledevice}, or embedded systems, rather than in
		centralized \gls{data} centers \cite{Shi2016}. 
		In \gls{ml}, edge computing supports low-latency \gls{inference} and reduced
		communication by executing parts of \gls{ml} \glspl{algorithm} on or near
		\gls{data}-generating \glspl{device} \cite{Nguyen2019}. 
		\\
		See also: \gls{cloudcomputing}, \gls{flsystem}, \gls{mlsystem}.},
	first={\foreignlanguage{greek}{υπολογιστική ακμής}},
	text={\foreignlanguage{greek}{υπολογιστική ακμής}},
	type=mlsystems,
	user1={\foreignlanguage{greek}{υπολογιστική ακμής}}, %nominative
	user2={\foreignlanguage{greek}{υπολογιστικής ακμής}}, %genitive
	user3={\foreignlanguage{greek}{υπολογιστική ακμής}} %accusative
}

\newglossaryentry{cloudcomputing}
{name={\foreignlanguage{greek}{υπολογιστική νέφους}},
	description={\foreignlanguage{greek}{Η υπολογιστική νέφους}\index{\foreignlanguage{greek}{υπολογιστική νέφους}} 
		(cloud computing) \foreignlanguage{greek}{είναι ένα υπολογιστικό παράδειγμα στο οποίο οι 
		υπολογιστικοί πόροι όπως η επεξεργασία, η αποθήκευση, και η δικτύωση παρέχονται ως 
		υπηρεσίες κατ' αίτηση μέσω ενός δικτύου επικοινωνίας} 
		\cite{ComerCloudComputing2021}, \cite{DistributedSystems}, \cite{Armbrust2010}. 
		\foreignlanguage{greek}{Στη} \glsentryuseriii{ml}, \foreignlanguage{greek}{τα συστήματα 
		υπολογιστικής νέφους χρησιμοποιούνται συχνά για τη φιλοξενία μεγάλων} \glsentryuserv{dataset} 
		\foreignlanguage{greek}{και την εκτέλεση} \glsentryuserv{algorithm} \glsentryuserii{ml}. 
		\foreignlanguage{greek}{Σε αντίθεση με τα} \glsentryuservi{flsystem}, 
		\foreignlanguage{greek}{η υπολογιστική νέφους συνήθως συγκεντρώνει τα}
		\glsentryuserii{data} \foreignlanguage{greek}{και τον υπολογισμό εντός κέντρων} 
		\glsentryuserii{data} \foreignlanguage{greek}{που διαχειρίζεται ο πάροχος}.
		\\
		\foreignlanguage{greek}{Βλέπε επίσης:} \gls{ml}, \gls{dataset}, \gls{algorithm}, \gls{flsystem}, \gls{data}, \gls{mlsystem}.},
	first={\foreignlanguage{greek}{υπολογιστική νέφους}},
	text={\foreignlanguage{greek}{υπολογιστική νέφους}},
	type=mlsystems,
	user1={\foreignlanguage{greek}{υπολογιστική νέφους}}, %nominative
	user2={\foreignlanguage{greek}{υπολογιστικής νέφους}}, %genitive
	user3={\foreignlanguage{greek}{υπολογιστική νέφους}} %accusative
}

\newglossaryentry{checkpoint}
{name={checkpoint},
	description={TBC.\index{checkpoint} },
	first={checkpoint},
	type=mlsystems,
	text={checkpoint}
}

\newglossaryentry{checkpointing}
{name={checkpointing},
	description={TBC.\index{checkpointing} },
	first={checkpointing},
	type=mlsystems,
	text={checkpointing}
}

\newglossaryentry{crosssectionaldata}
{name={cross-sectional data},
	description={TBC.\index{cross-sectional data} },
	first={cross-sectional data},
	type=mlsystems,
	text={cross-sectional data}
}

\newglossaryentry{earlyexit}
{name={early exit (deep learning)},
	description={TBC.\index{early exit (deep learning)} }, 
	first={early exit},
	type=mlsystems,
	text={early exit}
}

\newglossaryentry{mlpipeline}
{name={machine learning pipeline (ML pipeline)},
	description={TBC.\index{machine learning pipeline (ML pipeline)} },
	first={machine learning pipeline (ML pipeline)},
	type=mlsystems,
	text={ML pipeline}
}

\newglossaryentry{spotinstance}
{name={spot instance},
	description={TBC.\index{spot instance} }, 
	first={spot instance},
	type=mlsystems,
	text={spot instance}
}

%federated learning (system view, not algorithmic variants)

%parameter server

%client–server architecture

%orchestration

%asynchronous updates

%data locality

%deployment

%inference service

%scalability

%resource heterogeneity