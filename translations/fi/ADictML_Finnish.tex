% !TeX spellcheck = en_US

\documentclass[journal,12pt,onecolumn]{article}
\usepackage[finnish]{babel}
\usepackage{csvsimple}
\usepackage{filecontents} 
\usepackage{qrcode}
\usepackage{../../assets/pdfplots}
\usepackage{amsthm}
\usepackage{amsmath,graphicx}
\DeclareMathOperator*{\aargmin}{arg\,min}
\usepackage{booktabs} 
\usepackage{algorithm, algpseudocode}
\usepackage{imakeidx}
\usepackage[utf8]{inputenc}  % Legacy engine compatibility
\usepackage[T1]{fontenc}
\newcommand\Square[1]{+(-#1,-#1) rectangle +(#1,#1)}
\usepackage{hyperref}       % hyperlinks

% ---- Speed toggle ----
\newif\iffastdraft
\fastdraftfalse      % set to \fastdraftfalse for the final build


%\usepackage[automake,stylemods={tree}]{glossaries-extra}
%\usepackage[automake=false]{glossaries-extra}
  
\iffastdraft
 \usepackage[automake=false]{glossaries-extra}
\else
 \usepackage[automake,stylemods={tree}]{glossaries-extra}
\fi
 

\usepackage{etoolbox}
\usetikzlibrary{arrows.meta,positioning,calc,fit,backgrounds}

%Define a new custom key for tagging entries
%\glsaddkey{tag}{}{\glstag}  % define the custom field "tag"
%Define explicitly named glossaries
%\newglossary*{main}{Main Glossary}
\newglossary*{basic}{Basic Terms}
\newglossary*{math}{Tools}
\newglossary*{systems}{ML Systems}
\newglossary*{advanced}{Advanced Terms}


\usepackage[utf8]{inputenc} % allow utf-8 input
\usepackage[T1]{fontenc}    % use 8-bit T1 fonts
\usepackage{url}            % simple URL typesetting
\usepackage{booktabs}       % professional-quality tables
\usepackage{amsfonts}       % blackboard math symbols   
\usepackage{bm}
\usepackage{nicefrac}       % compact symbols for 1/2, etc.
\usepackage{microtype}      % microtypography
\usepackage{xcolor}         % colors
\usepackage{tikz}
\usetikzlibrary{matrix,positioning,fit}
\usepackage{cite}
\usepackage{IEEEtrantools}
\usepackage{pgfkeys,pgfcalendar}
\usetikzlibrary{shapes}
\usetikzlibrary{shadows}
\usepackage{helvet}
\usepackage{float}
\usepackage{enumitem}
\usepackage{caption}
\usepackage{subcaption}
\captionsetup[figure]{name=Fig., labelsep=period} % Customize figure caption format
\renewcommand{\thetable}{\Roman{table}} % Use Roman numerals


% New packages
\usepackage{titling} % Better title
%\usepackage[style=ieee,sorting=none,citetracker,pagetracker=page,backend=biber,backref=true]{biblatex} % Proper bibliography
%\usepackage[numbib,nottoc]{tocbibind} % Bibliography to ToC

%\usepackage{silence}
%\WarningFilter{latex}{Empty bibliography}
%\WarningFilter{latex}{Citation}

%\addbibresource{../../assets/Literature.bib}


\usepgfplotslibrary{fillbetween}
\usetikzlibrary{intersections}


\usepackage{xspace}
\renewcommand{\baselinestretch}{1.5}
\usepackage{tikz}
\usetikzlibrary{positioning,trees}
\usetikzlibrary{shapes.geometric}
\usetikzlibrary{arrows.meta}
\usetikzlibrary{calc}
\usetikzlibrary{shapes.misc}
\usetikzlibrary{trees, arrows.meta}
\tikzstyle{mybox}=[draw=red,very thick,
rectangle, rounded corners, inner sep=10pt, inner ysep=20pt]
\tikzstyle{fancytitle}=[fill=red, text=black]

\tikzset{
	treenode/.style={shape=rectangle, rounded corners,
	draw, align=center,
	top color=white, bottom color=blue!20},
	root/.style={treenode, font=\Large, bottom color=red!30},
	env/.style={treenode, font=\ttfamily\normalsize},
	dummy/.style={circle,draw}
}


\definecolor{aaltoBlue}{RGB}{0, 51, 102} % Example color, adjust to match Aalto's brand


\newtheorem{assumption}{Assumption}
\newcommand{\scdots}[2][]{\mathinner{#1\overset{#2}{\cdots}#1}}
\definecolor{pinegreen}{cmyk}{0.92,0,0.59,0.25}
\definecolor{royalblue}{cmyk}{1,0.50,0,0}
\definecolor{lavander}{cmyk}{0,0.48,0,0}
\definecolor{violet}{cmyk}{0.79,0.88,0,0}

\tikzstyle{ncyan}=[circle, draw=cyan!70, thin, fill=white, scale=0.8, font=\fontsize{11}{0}\selectfont]
\tikzstyle{ngreen}=[circle,  draw=green!70, thin, fill=white, scale=0.8, font=\fontsize{11}{0}\selectfont]
\tikzstyle{nred}=[circle, draw=red!70, thin, fill=white, scale=0.8, font=\fontsize{11}{0}\selectfont]
\tikzstyle{ngray}=[circle, draw=gray!70, thin, fill=white, scale=0.55, font=\fontsize{14}{0}\selectfont]
\tikzstyle{nyellow}=[circle, draw=yellow!70, thin, fill=white, scale=0.55, font=\fontsize{14}{0}\selectfont]
\tikzstyle{norange}=[circle,  draw=orange!70, thin, fill=white, scale=0.55, font=\fontsize{10}{0}\selectfont]
\tikzstyle{npurple}=[circle,draw=purple!70, thin, fill=white, scale=0.55, font=\fontsize{10}{0}\selectfont]
\tikzstyle{nblue}=[circle, draw=blue!70, thin, fill=white, scale=0.55, font=\fontsize{10}{0}\selectfont]
\tikzstyle{nteal}=[circle,draw=teal!70, thin, fill=white, scale=0.55, font=\fontsize{10}{0}\selectfont]
\tikzstyle{nviolet}=[circle, draw=violet!70, thin, fill=white, scale=0.55, font=\fontsize{10}{0}\selectfont]
\tikzstyle{qgre}=[rectangle, draw, thin,fill=green!20, scale=0.8]
\tikzstyle{rpath}=[ultra thick, red, opacity=0.4]
\tikzstyle{legend_isps}=[rectangle, rounded corners, thin,fill=gray!20, text=blue, draw]
\usetikzlibrary{positioning}
\newtheorem{proposition}{Proposition}%[section]
\newtheorem{theorem}{Theorem}%[section]
\newtheorem{definition}[theorem]{Definition}
\newtheorem{lemma}[theorem]{Lemma}
\newtheorem{example}[theorem]{Example}
\newtheorem{corollary}[theorem]{Corollary}
\definecolor{lightblue}{RGB}{173, 216, 230}

\newcounter{exercise}[section]
\newenvironment{exercise}[1][]{\refstepcounter{exercise}\par\medskip
	\textbf{Exercise~\theexercise. #1} \rmfamily}{\medskip}
\renewcommand{\theexercise}{\arabic{section}.\arabic{exercise}}

\newcommand*{\datefmt}[3]{%
	\number#3~\pgfcalendarmonthname{#2} \number#1%
}

\font\myfont=cmr12 at 30pt
\font\myfonta=cmr12 at 20pt
\font\myfontb=cmr12 at 17pt
\font\myfontc=cmr12 at 15pt

\makeglossaries

\iffastdraft
\glsdisablehyper  % fewer hypertargets created
\else
\glsenablehyper
\fi

\input{../../assets/ml_macros.tex}


\iffastdraft
% \makeindex        % skip
\else
\makeindex
\fi

\input{ADictML_Glossary_Finnish.tex}

%\makeglossaries

\glsaddkey
{section}% key
{}% default value
{\glsentryalternative}% no link cs
{\Glsentryalternative}% no link ucfirst cs
{\glsalternative}% link cs
{\Glsalternative}% link ucfirst cs
{\GLSalternative}% link all caps cs

% References formatting fix
\makeatletter
\renewenvironment{thebibliography}[1]
     {\section*{\refname}
      \begin{list}{[\arabic{enumi}]}
        {\settowidth\labelwidth{[99]}
         \setlength{\labelsep}{1em}  % Space between number and text
         \setlength{\leftmargin}{2em}  % Align numbers with section title
         \setlength{\itemindent}{0pt}
         \setlength{\listparindent}{0pt}
         \usecounter{enumi}}
         \sloppy\clubpenalty4000\widowpenalty4000
         \sfcode`\.=1000\relax}
     {\end{list}}
\makeatother

\begin{document}
\bstctlcite{bstctl:nodash}

\title {\vspace*{10mm}
	{\huge {\bf The {\fontsize{40}{48}\selectfont \textbf{\textsf{A\hspace*{-2mm}''}}}\hspace*{-4mm}allon \\ koneoppimisen sanakirja}}  \\[-5mm] 
	% \small Comments are warmly welcome at \url{alex.jung@aalto.fi}.
}

%\title {\vspace*{10mm}
%	{\huge {\bf The {\fontsize{40}{48}\selectfont \textbf{\textsf{A\hspace*{-2mm}''}}}\hspace*{-4mm}alto \\ Dictionary of Machine Learning}}  \\[-5mm] 
	% \small Comments are warmly welcome at \url{alex.jung@aalto.fi}.
%}

%\author{\hspace{-2mm}Alexander Jung, \ Konstantina Olioumtsevits, \ and \ Juliette Gronier \\[-2mm]% <-this % stops a space
%\thanks{}
%}

\author{\hspace{-2mm}Alexander Jung${}^{1}$, \ Konstantina Olioumtsevits${}^{1}$, \ Ekkehard Schnoor${}^{1}$, \\[-2mm]
	Tommi Flores Ryynänen${}^{1}$,\ Juliette Gronier${}^{2}$ ja Salvatore Rastelli${}^{1}$ \\[-2mm]
	Käännös Mikko Seesto${}^{1}$, Janita Thusberg${}^{1}$ ja Helli Manninen${}^{1}$ \\[-2mm]
	${}^{1}$Aalto-yliopisto \quad ${}^{2}$ENS Lyon
}

%\predate{\begin{center} {\large Käännös Mikko Seesto, Janita Thusberg ja Helli Manninen} \end{center} \\[5mm]}

\maketitle


\begin{center}
	\resizebox{4cm}{!}{\qrcode{https://github.com/AaltoDictionaryofML/AaltoDictionaryofML.github.io}}\\[8mm]
	{\large	please cite as: A.\ Jung, K.\ Olioumtsevits, E.\ Schnoor, T.\ Ryynänen, J.\ Gronier ja S.\ Rastelli, \textit{Aallon koneoppimisen sanakirja}. Käännös J.\ Thusberg, H.\ Manninen ja M.\ Seesto. Espoo, Suomi: Aalto-yliopisto, 2025.}
\end{center}

\newpage 
\noindent{\bf\Large Acknowledgment}\

\noindent This dictionary of machine learning evolved through the development 
and teaching of several courses, including CS-E3210 Machine Learning: Basic Principles, 
CS-C3240 Machine Learning, CS-E4800 Artificial Intelligence, CS-EJ3211 Machine Learning with Python, 
CS-EJ3311 Deep Learning with Python, CS-E4740 Federated Learning, and 
CS-E407507 Human-Centered Machine Learning. These courses were offered at 
Aalto University \url{https://www.aalto.fi/en}, to adult learners via 
The Finnish Institute of Technology (FITech) \url{https://fitech.io/en/}, and to international 
students through the European University Alliance Unite! \url{https://www.aalto.fi/en/unite}.

\noindent We are grateful to the students who provided valuable feedback that helped shape this dictionary. 
Special thanks to Mikko Seesto for his meticulous proofreading.

\noindent This work was supported by
\begin{itemize} 
	\item the Research Council of Finland (grant 331197, 363624, 349966);
	\item the European Union (grant 952410);
	\item the Jane and Aatos Erkko Foundation (grant A835); 
	\item Business Finland, as part of the project Forward-Looking AI Governance in Banking and Insurance (FLAIG).
\end{itemize} 

\newpage
\tableofcontents

\newpage 
\input{ListSymbols_Finnish.tex}

\newpage

% Set up alphabetical headers for each letter section
%\printglossaries[nonumberlist]
\glsaddallunused


\printglossary[type=math,title={Tools}, nonumberlist]

\newpage
\printglossary[title={Machine Learning Concepts}, nonumberlist]

\printglossary[type=systems, title={Machine Learning Systems}, nonumberlist]



\iffastdraft
% no index in draft
\else
% Suppress page numbers on index pages
\newpage
\pagenumbering{gobble}
\pagestyle{empty}
\printindex
\fi


\newpage
% The bibliography seems to not work for some reason
\bibliographystyle{IEEEtran}
\bibliography{../../assets/Literature.bib}
%\printbibliography[heading=bibintoc]

\end{document}