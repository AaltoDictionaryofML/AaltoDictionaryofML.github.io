%Health Care
\newglossaryentry{survivalanalysis}
{name={survival analysis},
 description={A \index{survival analysis} survival analysis refers to \gls{ml} applications 
              involving time-to-event \gls{data}. The goal is to predict the time until an 
               event of interest (e.g., failure, death, or relapse) occurs \cite{Wang2019_2025}. \\
    See also: \gls{hazardfunction}, \gls{coxph}, \gls{rsf}.
  },
  first={survival analysis},
  type=healthcare,
  text={survival analysis}
}

\newglossaryentry{rsf} 
{name={random survival forest},
  description={A\index{random survival forest} random survival forest (RSF) 
        \cite{Ishwaran2008} is an ensemble of \glspl{decisiontree} designed for 
        time-to-event (survival) \gls{data}. Each tree is grown on a \gls{bootstrap} 
        sample of the original \gls{dataset} with random \gls{feature} selection 
        at splits, using survival-specific split criteria (e.g., log-rank). 
       The forest aggregates per-tree survival estimates to produce an overall 
       survival function and risk scores, and naturally handles right-censoring.
       RSF extends the principles of \gls{randomforest} to survival settings and 
       has been shown to perform well in high-dimensional biomedical and real-world \glspl{dataset}
    \cite{Wang2019,Salerno2023}.
    \\
    See also: \gls{decisiontree}, \gls{dataset}, \gls{randomforest}.},
  first={random survival forest (RSF)},
  type=healthcare,
  text={RSF}
}

\newglossaryentry{hazardfunction}
{name={hazard function},
  description={A\index{hazard function} hazard function, also known as the 
               hazard rate or failure rate, represents the immediate risk 
               of an event occurring at a specific time, given that the 
               subject has survived up to that time. 
    \\
    See also: \gls{survivalanalysis}, \gls{coxph}.},
  first={hazard function},
  type=healthcare,
  text={hazard function}
}


\newglossaryentry{survivaloutcome}
{name={survival outcome}, 
 plural={survivaloutcomes},
 description={A \index{survival outcome} survival outcome refers to 
               the observed time-to-event information used as the \gls{target} 
               variable in \gls{survivalanalysis}, typically represented by a 
               pair consisting of an event time and an event indicator. The event 
               time denotes the duration from a defined origin (e.g., diagnosis 
               or treatment initiation) to the occurrence of an event of interest, 
               while the indicator specifies whether the event was observed or the 
               observation was censored. survival outcome form the basis for estimating 
               survival functions, hazard rates, and risk scores in \glspl{model} 
               such as the \gls{coxph} \gls{model} \cite{Cox1972,Kleinbaum2012}.
  \\
  See also: \gls{survivalanalysis}, \gls{hazardfunction}, \gls{covariate}, \gls{indicator}.},
  first={survival outcome},
  type=healthcare,
  text={survival outcome}
}


\newglossaryentry{coxph}
{name={Cox Proportional Hazards},
  description={A\index{Cox Proportional Hazards} Cox Proportional Hazards 
               (CoxPH) \gls{model} is a semi-\gls{parameter} \gls{model} 
               used in \gls{survivalanalysis} to relate multiple \glspl{covariate} 
               to the hazard rate of an event. It assumes that the \gls{lossfunc} 
               for an individual is the product of a \gls{baseline} hazard and an 
               exponential function of \glspl{covariate}. The \gls{model} estimates 
               relative \gls{risk} without requiring the \gls{baseline} hazard to be specified explicitly.
    \\
    See also: \gls{survivalanalysis}, \gls{hazardfunction}, \gls{covariate}.},
  first={Cox Proportional Hazards (CoxPH)},
  type=healthcare,
  text={CoxPH}
}

\newglossaryentry{coxnet}
{name={Cox Neural Network},
  description={A\index{Cox Neural Network} Cox Neural Network (CoxNet) is a 
               \gls{deeplearning} extension of the \gls{coxph} \gls{model} 
               that uses \glspl{ann} to learn complex, non-linear relationships 
               between \glspl{covariate} and survival outcomes, typically 
               represented by event times and censoring indicators. 
                It optimizes the Cox partial likelihood function while allowing 
                for flexible \gls{feature} representations, making it suitable 
                for high- \gls{dimension} or unstructured \gls{data} such as images or text.
    \\
    See also: \gls{coxph}, \gls{survivalanalysis}, \gls{covariate}, \gls{ann}.},
  first={Cox Neural Network (CoxNet)},
  type=healthcare,
  text={CoxNet}
}

\newglossaryentry{baselinehazard}
{name={baseline hazard},
  description={A\index{baseline hazard} \gls{baseline} hazard, denoted \(h_{0}(t)\), represents the underlying 
    \gls{hazardfunction} for an individual with \gls{baseline} or reference \glspl{feature} (typically coded as zero) 
    in the \gls{coxph} \gls{model}. It captures how the event risk changes over time independently of \gls{covariate} effects. 
    Although the \gls{coxph} \gls{model} does not specify a parametric form for \(h_{0}(t)\), it can be estimated 
    non-parametrically from the \gls{data} using methods such as the Breslow estimator \cite{Kleinbaum2012}. 
    \\
    See also: \gls{hazardfunction}, \gls{covariate}, \gls{coxph}, \gls{survivalanalysis}.},
  first={baseline hazard},
  type=healthcare,
  text={baseline hazard}
}